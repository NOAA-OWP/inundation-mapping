%%%%%%%%%%%%%%%%%%%%%%%%%%%%%%%%%%%%%%%%%%%%%%%%%%%%%%%%%%%%%%%%%%%%%%%%%%%%
% AGUJournalTemplate.tex: this template file is for articles formatted with LaTeX
%
% This file includes commands and instructions
% given in the order necessary to produce a final output that will
% satisfy AGU requirements, including customized APA reference formatting.
%
% You may copy this file and give it your
% article name, and enter your text.
%
%
% Step 1: Set the \documentclass
%
%

%% To submit your paper:
\documentclass[draft]{dependencies/agujournal2019}
\usepackage{url} %this package should fix any errors with URLs in refs.
\usepackage{lineno}
\usepackage[inline]{dependencies/trackchanges} %for better track changes. finalnew option will compile document with changes incorporated.
\usepackage{soul}
\usepackage{rotating}

%% added
\usepackage{multirow}
\usepackage{amsmath}
\usepackage{float}

\linenumbers

%%%%%%%
% As of 2018 we recommend use of the TrackChanges package to mark revisions.
% Add track changes editors
\addeditor{Fernando Aristizabal}

% The trackchanges package adds five new LaTeX commands:
%
%  \note[editor]{The note}
%  \annote[editor]{Text to annotate}{The note}
%  \add[editor]{Text to add}
%  \remove[editor]{Text to remove}
%  \change[editor]{Text to remove}{Text to add}
%
% complete documentation is here: http://trackchanges.sourceforge.net/
%%%%%%%

\drafttrue

%% Enter journal name below.
%% Choose from this list of Journals:
%
% JGR: Atmospheres
% JGR: Biogeosciences
% JGR: Earth Surface
% JGR: Oceans
% JGR: Planets
% JGR: Solid Earth
% JGR: Space Physics
% Global Biogeochemical Cycles
% Geophysical Research Letters
% Paleoceanography and Paleoclimatology
% Radio Science
% Reviews of Geophysics
% Tectonics
% Space Weather
% Water Resources Research
% Geochemistry, Geophysics, Geosystems
% Journal of Advances in Modeling Earth Systems (JAMES)
% Earth's Future
% Earth and Space Science
% Geohealth
%
% ie, \journalname{Water Resources Research}

\journalname{Water Resources Research}


% starts document
\begin{document}


%% Title and Authors
%% ------------------------------------------------------------------------ %%
%  Title
%
% (A title should be specific, informative, and brief. Use
% abbreviations only if they are defined in the abstract. Titles that
% start with general keywords then specific terms are optimized in
% searches)
%
%% ------------------------------------------------------------------------ %%

% Example: \title{This is a test title}

\title{Reducing Horton-Strahler Stream Order Can Enhance Flood Inundation Mapping Skill with Applications for the U.S. National Water Model}

%% ------------------------------------------------------------------------ %%
%
%  AUTHORS AND AFFILIATIONS
%
%% ------------------------------------------------------------------------ %%

% Authors are individuals who have significantly contributed to the
% research and preparation of the article. Group authors are allowed, if
% each author in the group is separately identified in an appendix.)

% List authors by first name or initial followed by last name and
% separated by commas. Use \affil{} to number affiliations, and
% \thanks{} for author notes.
% Additional author notes should be indicated with \thanks{} (for
% example, for current addresses).

% Example: \authors{A. B. Author\affil{1}\thanks{Current address, Antartica}, B. C. Author\affil{2,3}, and D. E.
% Author\affil{3,4}\thanks{Also funded by Monsanto.}}


\authors{Fernando Aristizabal\affil{1,2,3}, Fernando Salas\affil{3}, Brian Avant\affil{1,3}, Bradford Bates\affil{1,3}, Trevor Grout\affil{1,3}, Gregory Petrochenkov\affil{4}, Nick Chadwick\affil{1,3}, Zachary Wills\affil{3,5}, Fred Ogden \affil{3}, Jasmeet Judge\affil{2}}


\affiliation{1}{Lynker Technologies, Leesburg, VA, USA}
\affiliation{2}{Center for Remote Sensing, University of Florida, Gainesville, FL, USA}
\affiliation{3}{National Water Center, Office of Water Prediction, National Oceanic and Atmospheric Administration, Tuscaloosa, AL, USA}
\affiliation{4}{New York Water Science Center, United States Geological Survey, Troy, NY, USA}
\affiliation{5}{Cooperative Institute for Satellite Earth System Studies, University of Alabama, Tuscaloosa, AL, USA}

%(repeat as many times as is necessary)

%% Corresponding Author:
% Corresponding author mailing address and e-mail address:

% (include name and email addresses of the corresponding author.  More
% than one corresponding author is allowed in this LaTeX file and for
% publication; but only one corresponding author is allowed in our
% editorial system.)

% Example: \correspondingauthor{First and Last Name}{email@address.edu}

\correspondingauthor{Fernando Aristizabal}{fernando.aristizabal@noaa.gov}



%% Keypoints
%% Keypoints, final entry on title page.

%  List up to three key points (at least one is required)
%  Key Points summarize the main points and conclusions of the article
%  Each must be 140 characters or fewer with no special characters or punctuation and must be complete sentences

% Example:
% \begin{keypoints}
% \item	List up to three key points (at least one is required)
% \item	Key Points summarize the main points and conclusions of the article
% \item	Each must be 140 characters or fewer with no special characters or punctuation and must be complete sentences
% \end{keypoints}

\begin{keypoints}
\item The National Water Model produces forecasts at more than 2.8 million river reaches in the United States.
\item OWP FIM produces flood inundation maps by converting discharges to stages and stages to extents and depths.
\item FIM 3 is a modified version of the Height Above Nearest Drainage technique that provides version improvement with FIM 1 and 2.
\end{keypoints}



%% Abstract
% ------------------------------------------------------------------------ %%
%
%  ABSTRACT and PLAIN LANGUAGE SUMMARY
%
% A good Abstract will begin with a short description of the problem
% being addressed, briefly describe the new data or analyses, then
% briefly states the main conclusion(s) and how they are supported and
% uncertainties.

% The Plain Language Summary should be written for a broad audience,
% including journalists and the science-interested public, that will not have 
% a background in your field.
%
% A Plain Language Summary is required in GRL, JGR: Planets, JGR: Biogeosciences,
% JGR: Oceans, G-Cubed, Reviews of Geophysics, and JAMES.
% see http://sharingscience.agu.org/creating-plain-language-summary/)
%
%% ------------------------------------------------------------------------ %%

%% \begin{abstract} starts the second page

\begin{abstract}
Height Above Nearest Drainage (HAND), a drainage normalizing terrain index, is a means able of producing flood inundation maps (FIMs) from the National Water Model (NWM) at large scales and high resolutions using reach-averaged synthetic rating curves. 
We highlight here that HAND is limited to producing inundation only when sourced from its nearest flowpath, thus lacks the ability to source inundation from multiple fluvial sources.
A version of HAND, known as Generalized Mainstems (GMS), is proposed that discretizes a target stream network into segments of unit Horton-Strahler stream order known as level paths (LP).
The FIMs associated with each independent LP are then mosaiced together, effectively turning the stream network into discrete groups of homogeneous unit stream order by removing the influence of neighboring tributaries.
Improvement in mapping skill is observed by significantly reducing false negatives at river junctions when the inundation extents are compared to FIMs from that of benchmarks.
A more marginal reduction in the false alarm rate is also observed due to a shift introduced in the stage-discharge relationship by increasing the size of the catchments.
We observe that the improvement of this method applied at 4-5\% of the entire stream network to 100\% of the network is about the same magnitude improvement as going from no drainage order reduction to 4-5\% of the network.
This novel contribution is framed in a new open-source implementation that utilizes the latest combination of hydro-conditioning techniques to enforce drainage and counter limitations in the input data.
\end{abstract}
%
\section*{Plain Language Summary}
Flooding is one of the most impactful natural disasters on life and property. 
The United States National Water Model (NWM) provides flood forecasts to adequately warn people for safe evacuations and protective measures across the entire country.
To convert streamflow from the NWM to flood inundation maps (FIM), a model, Height Above Nearest Drainage (HAND), is used that translates elevation data from height above mean sea-level to height above the nearest river.
This model suffers from issues in mapping performance because inundation sourced from rivers is only considered from the nearest river.
We developed a technique that mitigates these errors by removing consideration for neighboring tributaries in the relative elevation computation process.
This is done by splitting the stream network into continuous river segments known as level paths (LPs) which removes the effects of tributaries.
HAND is computed independently for each LP and the resulting FIMs are mosaiced together to form one seamless map.
By computing HAND and catchments on an LP scale, catchments and FIMs are allowed to overlap which can account for multiple river sources of inundation especially at river confluences.
We compared these HAND derived FIMs to maps from physically-based models and found improvement in mapping performance.
%



%%% Suggested section heads:
% \section{Introduction}
%
% The main text should start with an introduction. Except for short
% manuscripts (such as comments and replies), the text should be divided
% into sections, each with its own heading.

% Headings should be sentence fragments and do not begin with a
% lowercase letter or number. Examples of good headings are:

% \section{Materials and Methods}
% Here is text on Materials and Methods.
%
% \subsection{A descriptive heading about methods}
% More about Methods.
%
% \section{Data} (Or section title might be a descriptive heading about data)
%
% \section{Results} (Or section title might be a descriptive heading about the
% results)
%
% \section{Conclusions}

% Introduction
%%%%%%%%%%%%%%%%%%%%%%%%%%%%%%%%%%%%%%%%%%%%%%%%%%%%%%%%
\section{Introduction}
%%%%%%%%%%%%%%%%%%%%%%%%%%%%%%%%%%%%%%%%%%%%%%%%%%%%%%%%
%
Flooding is one of the most significant natural disasters in the United States (US) affecting both the loss of life and property. 
In 2017 and 2019, river and flash flooding combined represented the leading cause of death and the second leading cause in 2018 among all natural disasters in the US \cite{national_weather_service_2020,national_weather_service_2019,national_weather_service_2018}. 
More than an average of 104 deaths per year are attributed to flood events from the 10 year period ending in 2019 \cite{us_department_of_commerce_2020}. 
With respect to property damages, river and flash flooding have contributed to 60.7, 1.6, and 3.7 billion non-inflation adjusted US dollars in the annual periods of 2017 to 2019, respectively \cite{national_weather_service_2020,national_weather_service_2019,national_weather_service_2018} with the large spike in 2017 attributed to the Hurricane Harvey event along the Gulf coast. 
Unencouragingly, the trends related to flood damages and fatalities have been steadily increasing over recent decades. \cite{mallakpour2015changing,downton2005reanalysis,kunkel1999temporal,pielke2000precipitation,corringham2019effect}. 
Some are expecting that the hydrologic cycle will intensify which will lead to more extreme precipitation in some areas along with a greater risk of flooding \cite{tabari2020climate,milly2002increasing,wing2018estimates}. 
Increasing trends in frequency and risk are not uniform across spatial regions with work by \citeA{slater2016recent} indicating that trends are increasing across the US Midwest/Great Lakes region while decreasing in coastal Southeast, Southwest and California.
%
%%%%%%%%%%%%%%%%%%%%%%%%%%%%%%%%%%%%%%%%%%%%%%%%%%%%%%%%
\subsection{Operational Forecasting}
%%%%%%%%%%%%%%%%%%%%%%%%%%%%%%%%%%%%%%%%%%%%%%%%%%%%%%%%
%
Operational flood forecasting systems are primary tools in developing accurate forecasts for public awareness prior to life or property damaging events occur. 
One of these operational systems is the Advanced Hydrologic Prediction System (AHPS) maintained by National Oceanic Atmospheric Administration (NOAA) National Weather Service (NWS) with more than 3,000 forecast points across the US at typically short forecast horizons of 24 or 72 hours \cite{mcenery2005noaa}.
AHPS provides forecasting services in the form of ensemble stream flows at more than 3,000 forecast points and flood inundation maps (FIM) at more than 150 of those locations shown in Figure \ref{fig:forecast_points}.
Besides the AHPS forecast points, we display in Figure \ref{fig:forecast_points} two forecasting networks relevant to the National Water Model (NWM) which will be introduced in Section \ref{ssec:national_water_model}.
AHPS implements a series of advances including model calibration techniques \cite{zhang2003hydrologic,hogue2003multi,duan2003global,gupta2003advances,parada2003multi}, distributed modeling approaches \cite{reed2004overall,koren2004hydrology,duan2002results}, ensemble forecasting \cite{day1985extended,seo2000simulation,mullusky2002simplified,herr2002simplified}, enhanced data analysis procedures \cite{mcenery2005noaa}, flood-forecasting inundation maps \cite{cajina2002fldview}, hydraulic routing models \cite{fread1973technique,cajina2002fldview}, and multisensor precipitation techniques \cite{breidenbach1999accounting,kondragunta2001outlier,seo2002real,bonnin1996noaa}.
On an approximate basis, there is only one forecast point every 1,450 km of river and one forecast point with FIM every 29,000 km.
Despite the AHPS advances in operational flood forecasting, it lacks sufficient spatial coverage and long-range forecast horizons.
%
\begin{figure}[h!]
\centering
\includegraphics[scale=1.0]{figures/forecast_points.jpg}
\caption{Forecast points with and without FIM in United States' Advanced Hydrologic Prediction System. Also show are the National Water Model stream networks at the full resolution (FR) and Replace and Reroute (RnR) Mainstems (MS) resolution.}
\label{fig:forecast_points}
\end{figure}
%
%%%%%%%%%%%%%%%%%%%%%%%%%%%%%%%%%%%%%%%%%%%%%%%%%%%%%%%%
\subsection{National Water Model}
\label{ssec:national_water_model}
%%%%%%%%%%%%%%%%%%%%%%%%%%%%%%%%%%%%%%%%%%%%%%%%%%%%%%%%
%
Additional work is required to fill-in the gaps that the AHPS leaves in terms of spatial and temporal coverage.
To broaden the forecasting domain, the Office of Water Prediction (OWP) at the National Water Center (NWC) in Tuscaloosa, Alabama commissioned the development of the National Water Model (NWM) which is an instance of the Weather Research and Forecast Hydrologic Model (WRF-Hydro) \cite{gochis2018wrf,cosgrove2019evolution}. 
The NWM forecasts river discharges at more than 2.7 million forecast points at a variety of time horizons including some medium (10 day) and long (30 day) range forecast horizons.
The NWM enhances but does not replace the spatial and temporal domain of the current AHPS capabilities at the 13 River Forecast Centers (RFC) in areas known as `hydro-blind'. 
It's simply an additional model to be used in the forecasting and early warning decision making.
Furthermore, the NWM as of V1.2 has implemented not only assimilation of real-time United States Geological Survey (USGS) stage discharges but also assimilation of flow forecasts from the AHPS forecast points which are in turn routed downstream and updated whenever a new point is reached. 
This configuration of the NWM is known as `Replace and Route' or `RnR' and is used to enhance the forecasting skill of the NWM with the best available regional-scale data.
Figure \ref{ssec:national_water_model} shows Full-Resolution (FR) modeling stream network as well as the RnR mainstems (MS) network.

The National Hydrography Dataset Plus (NHDPlus) V2.1 is the basis for the hydrofabric in the NWM due to its comprehensive use with the hydrologic communities' stakeholders \cite{mckay2012nhdplus}. 
The term hydrofabric is used within the NWM jargon to describe the subset of hydrography comprised of geospatial datasets required for hydrologic modeling including but not limited to stream networks, catchments, channel properties, and elevation data. 
The Muskingam-Cunge routing method is used within the NWM to reduce computational requirements of a continental scale model \cite{bedient2008hydrology,ponce1994variable,gochis2018wrf}.
Muskingam-Cunge routing scheme has been demonstrated by \citeA{cunge1969subject} to be equivalent to the convective-diffusive wave method without consideration to wave dampening.
As a result of high computational costs and large spatial domains, the need for high-resolution FIM at 10m or better requires additional post-processing from the principal output of the NWM which is forecast river discharges at the reach scale. The Height Above Nearest Drainage (HAND) terrain model is one such technique that can be used, along with synthetic rating curves (SRC), to convert riverine discharges to stages to inundation extents.
%
%%%%%%%%%%%%%%%%%%%%%%%%%%%%%%%%%%%%%%%%%%%%%%%%%%%%%%%%
\subsection{Height Above Nearest Drainage}
%%%%%%%%%%%%%%%%%%%%%%%%%%%%%%%%%%%%%%%%%%%%%%%%%%%%%%%%
%
HAND normalizes topography along the nearest drainage path and its been demonstrated to be a good proxy and indicator of a series of important environmental conditions including soil environments, landscape classes, soil gravitational potentials, geomorphologies, soil moisture, and ground water dynamics \cite{renno2008hand,nobre2011height}. 
\citeA{nobre2016hand} showed evidence for utilizing the drainage normalizing HAND dataset as a proxy for flood potential to make static flood inundation maps from known stages.
The terrain index as even gone on to provide additional utility in the observation of riverine flood inundation mapping from remote sensing especially in areas of high electromagnetic interference such as vegetated and anthropogenic areas \cite{aristizabal2020high,shastry2019using,huang2017comparison,twele2016sentinel}.
\citeA{zheng2018river} developed methodology for determining stage-discharge relationships known as synthetic rating curves (SRC) by sampling reach-averaged parameters from HAND datasets and inputting into the Manning's equation \cite{gauckler1867etudes,manning1890flow}.
This collection of methods, coupling HAND with SRCs, has been experimented and compared to other sources of FIM including engineering scale models, in-situ observation, and remote sensing based observation with solid results in large spatial scale applications \cite{godbout2019error,johnson2019integrated,garousi2019terrain,nobre2016hand,afshari2018comparison,zheng2018geoflood,teng2015rapid,teng2017flood,zhang2018comparative}.
%
%%%%%%%%%%%%%%%%%%%%%%%%%%%%%%%%%%%%%%%%%%%%%%%%%%%%%%%%
\subsection{HAND Implementations}
%%%%%%%%%%%%%%%%%%%%%%%%%%%%%%%%%%%%%%%%%%%%%%%%%%%%%%%%
%
Due to significant advances in high-performance computing (HPC) and large scale high-resolution DEM's such as the National Elevation Dataset (NED) at the 10m scale, HAND has been implemented into software for large-scale, continental computation. 
HAND was initially implemented into operational software by the National Flood Interoperability Experiment (NFIE) to generate FIM hydrofabric (will be used interchangeably with the datasets produced by HAND) rapidly on a high-performance computer (HPC) \cite{maidment2017conceptual,liu2016cybergis}. 
NFIE used open-source dependencies including the Terrain Analysis Using Digital Elevation Models (TauDEM) \cite{tarboton2005terrain} and the Geospatial Data Abstraction Library (GDAL) \cite{warmerdam2008geospatial} to compute HAND for the Continental United States (CONUS) at 331 Hydrologic Unit Code (HUC) 6 processing units in 1.34 CPU years.
By allocating 31 nodes at 20 cores per for a total of 620 available cores to the overall operation, it enabled the production to finish up in 36 hours consuming 3.2TB of peak memory and 5TB of total disk space.
Originally, NFIE utilized the National Hydrography Dataset (NHD) Plus Medium Resolution (MR) to etch or burn flowlines prior to further conditioning but more recent work has advanced this to the more current NHDPlus High Resolution (HR) \cite{liu2020height}. 
The original NFIE dataset was employed by the NWC to produce forecast FIM from the NWM for use within its network of RFC's for additional guidance in hydro-blind regions and tagged as OWP's FIM V1.0.
Further work by \citeA{djokic2019arc}, implemented a series of improvements to HAND including equidistant reaches, updates to use with NHDPlusHR hydrography, and AGREE-DEM reconditioning \cite{hellweger1997agree} into an ESRI Arc-Hydro workflow with use in ArcGIS and tagged as OWP FIM V2.0. 
More notably the software added the ability to derive drainage potentials on both the NWM FR and MS stream networks which leverages the lower drainage density and Horton-Strahler stream order of the MS network.
Overall, the software package is estimated to run CONUS at the full-resolution in 0.55 CPU years in a desktop setting.
In addition to the domestic work done in the US, some studies have expanded upon HAND to cover global domains at 30m resolutions \cite{yamazaki2019merit,donchyts2016global}.
%
%%%%%%%%%%%%%%%%%%%%%%%%%%%%%%%%%%%%%%%%%%%%%%%%%%%%%%%%
\subsection{OWP FIM 3.0 `Cahaba'}
%%%%%%%%%%%%%%%%%%%%%%%%%%%%%%%%%%%%%%%%%%%%%%%%%%%%%%%%
%
Many of those assessing HAND's efficacy for producing FIM have noted opportunities for improvement. 
\citeA{godbout2019error} found how reach length and slope are important parameters for maximizing mapping skill with the moderate values performing best. 
The colinearity of reach length and slope led \citeA{godbout2019error} to propose that reaches of extreme lengths performed worse because of the extreme slope values, a parameter directly represented in Manning's equation. 
Issues with the reach-average approaches have been documented in \citeA{tuozzolo2019impact} where large within reach variance of the roughness Manning's n coefficient have been observed.
Furthermore, \citeA{garousi2019terrain} noted improvements to mapping efficacy by conditioning monotonically decreasing thalweg elevations, adjusting the Manning's n roughness coefficient, and using higher resolution (3m) Digital Elevation Model's (DEM) derived from light detection and ranging (Lidar).
Use of higher resolution DEMs in that study was motivated by previous work with Lidar DEMs and least-cost thalweg derivations \cite{zheng2018geoflood}.
Further work by \citeA{johnson2019integrated} noted the general under-prediction of HAND and suggested tuning the Manning's n parameter to better align SRC's with observations. 
Additionally, the sensitivity to low topographic relief and channel slope have been observed \cite{johnson2019integrated,godbout2019error}. 
Most notably, HAND has been shown to demonstrate sensitivity to drainage network density known colloaquially as the catchment boundary problem \cite{zhang2018comparative,mcgehee2016modified,li2020evaluation,nobre2016hand}.
This sensitivity is noted at higher order streams with large flows and low Froude numbers and is manifested by a constriction in the inundation extents at the junction location.
These findings suggest improvements to HAND are required that utilize advanced computational algorithms and software to compute a FIM hydrofabric required for producing continental-scale FIM.

A study is proposed to introduce and evaluate the development of OWP FIM V3.0 `Cahaba' which utilizes some of the latest mapping techniques along with a few novel developments in a computationally friendly framework. 
Most notably this study demonstrates how FIM performance skill with HAND can be improved by reducing Horton-Strahler stream orders \cite{horton1945erosional,strahler1952hypsometric,strahler1952hypsometric} of the stream networks.
The following methods and results describe the work in more detail and demonstrates its efficacy in producing enhanced FIM for the NWM for operational forecast applications. 
Furthermore, it enables the contribution from the broader hydro-community through its use of open-source code as further advances are made in this developing area of research.
%

%
% Materials and Methods
\clearpage % this clears figures before references
\section{Materials and Methods}

\subsection{HAND}

\subsection{Synthetic Rating Curves}

\subsection{Evaluation and Testing}


%
% Results
\clearpage % this clears figures before references
%%%%%%%%%%%%%%%%%%%%%%%%%%%%%%%%%%%%%%%%%%%%%%%%%%%%%%%%
%%%%%%%%%%%%%%%%%%%%%%%%%%%%%%%%%%%%%%%%%%%%%%%%%%%%%%%%
\section{Results}
\label{sec:results}
%%%%%%%%%%%%%%%%%%%%%%%%%%%%%%%%%%%%%%%%%%%%%%%%%%%%%%%%
%%%%%%%%%%%%%%%%%%%%%%%%%%%%%%%%%%%%%%%%%%%%%%%%%%%%%%%%
%
%%%%%%%%%%%%%%%%%%%%%%%%%%%%%%%%%%%%%%%%%%%%%%%%%%%%%%%%
\subsection{Mapping Performance}
\label{ssec:mapping_performance}
%%%%%%%%%%%%%%%%%%%%%%%%%%%%%%%%%%%%%%%%%%%%%%%%%%%%%%%%
%
We produced FIMs for the entire BLE domain within the 49 HUC8 areas across several states in the south central US. 
The forecasted FIMs using the discharges for the 1\% (100 year) and 0.2\% (500 year) recurrence flows directly from HEC-RAS were used to avoid noise and errors from hydrological processes.
We computed the statistics (CSI, POD, and FAR) for both 100 and 500 year events for Mannings N set to 0.06 and 0.12. 
The distribution of these statistics can be examined in Figure \ref{fig:violin_plot} as violin plots.
Each half of a violin plot represents the kernel density estimation (KDE) for a given model (FR, MS, GMS), given Manning's n value (0.06, 0.12), and given recurrence interval (1\%, 0.2\%), and performance metric (CSI, POD, FAR).
We also denote trend lines for each metric and Manning's n setting as well as their respective slope estimate and one-tailed p-value denoting the level of significance of the trend.

Aggregating the metrics in the method above treats each HUC8 as it's own unit and does little to consider the size differences of the HUCs. 
In an opposing aggregation method, we illustrate in Table \ref{tab:aggregate_metrics}  the CSI, POD, and FAR recomputed for the entire domain using the sum of all the TPs, FPs, and FNs. 
%
\begin{figure}[h!]
\centering
\includegraphics[scale=0.9]{figures/violin_plots.jpg}
\caption{Shows kernel density estimation of the distributions (sample size = 49) for 1\% (100 year) and 0.2\% (500 year) along with horizontal, dashed lines for the 25th, 50th, and 75th percentiles (in order from bottom to top).
The sub-figures separate the combination of three metrics (CSI, POD, and FAR) for two settings of Manning's n (0.06 and 0.12).
Trend lines for each metric - Mannings combination are shown (sample size = 294) along with associated slope and p-value of slope testing one-tailed significance.}
\label{fig:violin_plot}
\end{figure}
%
\begin{table}[h!]
\caption{Recomputed CSI, POD, and FAR using the primary metrics, TPs, FPs, and FNs, aggregated for BLE domain.
         Best value for across models is highlighted in bold.}
\label{tab:aggregate_metrics}
\centering
%\begin{tabular}{|p{2cm}|p{2cm}|p{2cm}|p{2cm}|}
\begin{tabular}{|c|c||c|c|c|c|c|c|}
\hline
\multirow{2}{*}{Metric} & \multirow{2}{*}{Manning's n} & \multicolumn{2}{|c|}{FR} & \multicolumn{2}{|c|}{MS} & \multicolumn{2}{|c|}{GMS} \\
\cline{3-8}
  &  & 100yr & 500yr & 100yr & 500yr & 100yr & 500yr \\
\hline
\multirow{2}{*}{CSI} & 0.06 & 0.5576 & 0.5839 & 0.5717 & 0.5990 & \textbf{0.5796} & \textbf{0.6075} \\
\cline{2-8}
  & 0.12 & 0.5915 & 0.6149 & 0.6054 & 0.6288 & \textbf{0.6182} & \textbf{0.6435} \\
\hline
\multirow{2}{*}{POD} & 0.06 & 0.6354 & 0.6575 & 0.6524 & 0.6755 & \textbf{0.6633} & \textbf{0.6863} \\
\cline{2-8}
  & 0.12 & 0.7255 & 0.7446 & 0.7460 & 0.7648 & \textbf{0.7606} & \textbf{0.7810} \\
\hline
\multirow{2}{*}{FAR} & 0.06 & 0.1800 & 0.1609 & 0.1787 & \textbf{0.1589} & \textbf{0.1778} & \textbf{0.1589} \\
\cline{2-8}
  & 0.12 & 0.2379 & 0.2208 & 0.2374 & 0.2204 & \textbf{0.2324} & \textbf{0.2148} \\
\hline
\end{tabular}
\end{table}
%
% Interpretation of metrics
From Figure \ref{fig:violin_plot} and Table \ref{tab:aggregate_metrics}, we denote several meaningful trends. 
Using CSI as an overall proxy for skill of the FIM, we note that generally speaking the skill is correlated with a reduction of the stream orders of the processing units used for HAND.
In other words, the more we derive HAND on networks of unit drainage density and mosaic the resulting FIMs, the better those FIMs perform.
While this trend is evident for both sets of Manning's n values, the trend is slightly more significant for the higher value of 0.12.
Other trends related to this Figure include the general performance premium for 0.2\% events as opposed to lower 1\% events.
We also note how the higher Manning's n value enhances performance for both of these recurrence intervals across all models.

Dissecting the improvements and trends presented in the previous paragraph comes down mostly to improvement in POD or a reduction in absolute amount of FNs.
POD being the primary driver in skill enhancement is evident across models by comparing the slope of the POD lines with the slope of the FAR lines.
Even though aggregating metrics by HUC8 yields a statistically zero trend, one does see a slight reduction in FAR across models that reduce HAND's maximum stream order.
Additionally, we note that POD is a primary driver in enhancing performance across Manning's n values as well.
This significant improvement comes at a cost of false alarms as the FAR increases significantly across Manning's n values.
%
%%%%%%%%%%%%%%%%%%%%%%%%%%%%%%%%%%%%%%%%%%%%%%%%%%%%%%%%
\subsection{Computational Performance}
\label{ssec:compuational_performance}
%%%%%%%%%%%%%%%%%%%%%%%%%%%%%%%%%%%%%%%%%%%%%%%%%%%%%%%%
%
The NFIE experiments were able to produce HAND for 331 HUC6's in 1.34 CPU years \cite{liu2016cybergis} and estimates using work from \citeA{djokic2019arc} put producing HAND at the FR NWM at 0.55 CPU years. 
For our work, we were able to produce HAND at the full NWM resolution in 0.13 CPU years which represents a substantial speed-up compared to previous works.
For the MS resolution an additional, 0.05 CPU years is required on top of this bringing the total to about 0.18 CPU years to produce 2,188 HUC8s that span additional areas not covered in previous HAND versions including Hawaii and Puerto Rico.
GMS which generalizes HAND production to level path scale adds a significant amount of CPU time to the process bringing the estimate total to about 1.17 CPU years.
%

%
% Discussion
\clearpage % this clears figures before references
%%%%%%%%%%%%%%%%%%%%%%%%%%%%%%%%%%%%%%%%%%%%%%%%%%%%%%%%
\section{Discussion}
\label{sec:discussion}
%%%%%%%%%%%%%%%%%%%%%%%%%%%%%%%%%%%%%%%%%%%%%%%%%%%%%%%%
%
Talk about calibration limitations. Cite Boulder and Mike J work.

Talk about cross-walking limitations, FEMA to NWM, NWM to FIM.

Fluvial limitations of HAND.

Limitations of FEMA stream network

Discuss how HAND dependence on Horton-Strahler stream order level and drainage density and how it improves with lower confluence points.


%
% Conclusions
\clearpage % this clears figures before references
%%%%%%%%%%%%%%%%%%%%%%%%%%%%%%%%%%%%%%%%%%%%%%%%%%%%%%%%
%%%%%%%%%%%%%%%%%%%%%%%%%%%%%%%%%%%%%%%%%%%%%%%%%%%%%%%%
\section{Conclusions}
\label{sec:conclusions}
%%%%%%%%%%%%%%%%%%%%%%%%%%%%%%%%%%%%%%%%%%%%%%%%%%%%%%%%
%%%%%%%%%%%%%%%%%%%%%%%%%%%%%%%%%%%%%%%%%%%%%%%%%%%%%%%%
%
Floods present a significant, underserved, and expanding risk to life, property, and resources.
Previous flood forecasting systems lacked the coverage to adequately inform society of these risks.
The National Water Model (NWM) developed by the National Oceanic and Atmospheric Administration's Office of Water Prediction, along with partners, provides increased spatial coverage and resolution as well as additional forecast time horizons on mostly hourly intervals.
Additional processing is required to convert streamflows from the NWM to river stages and finally to flood inundation maps (FIM).
Height Above Nearest Drainage (HAND) is a means of detrending digital elevations maps (DEM) by normalizing elevation to the nearest relevant drainage point.
HAND coupled with the use of reach averaged synthetic rating curves (SRC) provide such a means of creating continental scale FIM capabilities at high resolutions (1/3 arc-second, 10 m) and high temporal frequencies (up to 1 hr).
Scalable, open-source software was developed to produce HAND and associated datasets (catchments, SRCs, and cross-walking data) for the NWM forecasting area including Hawaii and Puerto Rico (https://github.com/NOAA-OWP/cahaba).
HAND is produced using the latest hydro-conditioning techniques to enforce monotonically decreasing elevations including stream burning, levee enforcement, pit-filling, stream channel excavation, thalweg breaching, headwater seeding, stream reach resampling, and more. 
Finally, we use this implementation to investigate the skill of the FIMs by varying the scale of the processing units used to derive HAND.
We illustrate that reducing the Horton-Strahler stream order of a HAND processing unit down to one enhances skill by significantly reducing false negatives at junctions of major streams.
This also affects the parameters used to compute stage-discharge relationships biasing discharge higher at given stages which reduced the rate of false positives.
FIM skill was evaluated over large spatial scales by comparsion to HEC-RAS 1D models.
Further investment in the SRC's is warranted by accounting for bathymetric errors inherited by the DEM and better accounting for localized friction values at varying flow magnitudes.
Utilizing the highest resolution Lidar and bathymetric data should also improve the vertical accuracy of HAND and better account for fine grain features that greatly affect inundation extents.
Due to inherent limitations with HAND, scalable physics-based methods will need to be worked on to provide better representation of flood extent dynamics in steady and unsteady conditions. 
%

%
% Acknowledgments
\clearpage % this clears figures before references
%%%%%%%%%%%%%%%%%%%%%%%%%%%%%%%%%%%%%%%%%%%%%%%%%%%%%%%%
%%%%%%%%%%%%%%%%%%%%%%%%%%%%%%%%%%%%%%%%%%%%%%%%%%%%%%%%
% Acknowledgments
\acknowledgments
%%%%%%%%%%%%%%%%%%%%%%%%%%%%%%%%%%%%%%%%%%%%%%%%%%%%%%%%
%%%%%%%%%%%%%%%%%%%%%%%%%%%%%%%%%%%%%%%%%%%%%%%%%%%%%%%%
%
This work was funded by the Office of Water Prediction (OWP) which is part of the National Oceanic and Atmospheric Administration's (NOAA) National Weather Service (NWS).
Lynker, under contract with OWP, facilitated this work and computational resources used in research and development.
We would like to thank some notable contributors of this work including Chief Scientist at OWP, Fred Ogden, for his technical expertise.
Additionally, David Blodgett from the United States Geological Survey (USGS) Water Mission Area was instrumental in helping define level paths and other hydrographic work.
More information on code availability, usage, and data retrieval for OWP FIM is available on GitHub and HydroShare \cite{inundationMapping2022,imHS2023}.
Thanks to the Earth and Space Science Informatics Partnership (ESIP) for storing data from this study for public use and dissemination helping to provide transparent datasets for further collaboration with the research community \cite{esipData2022}.

%
% Open Research
\clearpage
%%%%%%%%%%%%%%%%%%%%%%%%%%%%%%%%%%%%%%%%%%%%%%%%%%%%%%%%
%%%%%%%%%%%%%%%%%%%%%%%%%%%%%%%%%%%%%%%%%%%%%%%%%%%%%%%%
\openresearch
%%%%%%%%%%%%%%%%%%%%%%%%%%%%%%%%%%%%%%%%%%%%%%%%%%%%%%%%
%%%%%%%%%%%%%%%%%%%%%%%%%%%%%%%%%%%%%%%%%%%%%%%%%%%%%%%%
%
National Water Model (NWM) data used in this study includes the hydrofabric related datasets \cite{nwm2022hydrofabric} including catchments, streamlines, and reservoirs \cite{nwm2022hydrofabric}.
These are furnished by the National Oceanic and Atmospheric Administration (NOAA) Office of Water Prediction (OWP) via an Earth Science Information Partners (ESIP) Amazon Web Services (AWS) S3 Bucket \cite{esipData2022}.
OWP Flood Inundation Mapping (FIM) capabilities rely extensively on the use of the National Hydrography Plus High Resolution (NHDPlusHR) datasets including BurnLineEvents \cite{nhdplus2022vectors}, value-added attributes (VAA) \cite{nhdplus2022vectors}, water boundaries (WBD) or hydrologic unit code (HUC) geometries \cite{nhdplus2022wbd}, and digital elevation maps (DEM) \cite{nhdplus2022dems}.
Some additional datasets for processing include the National Levee Database (NLD) furnished by the United States Army Core of Engineers (USACE) \cite{engineers2016national}, Land-sea border from the Great Lakes Hydrography Dataset (GLHD) furnished by the Great Lakes Aquatic Habitat Framework (GLAHF) \cite{GreatLakesHydrographyDataset}, and a Land-sea border provided by OpenStreetMap (OSM) \cite{osm2021landsea}.
Evaluation data was furnished by Interagency Flood Risk Management (InFRM) consortium including cross-sections and flood depths \cite{fema2021base,fema2021estimated}.
Additionally, some FIM hydrofabric data including HAND grids, catchments, streamlines, synthetic rating curves, and cross-walk tables are available on the ESIP bucket \cite{esipData2022}.

Software used in preprocessing data, producing FIM hydrofrabic, generating FIM, computing metrics, and conducting analysis is available from a publicly available Github repository called ``inundation-mapping'' from the ``NOAA-OWP'' organization \cite{inundationMapping2022}.

%
% Bibliography
\clearpage % this clears figures before references
\bibliography{bibliography/owp_fim4_2022}
%
%%

%  Numbered lines in equations:
%  To add line numbers to lines in equations,
%  \begin{linenomath*}
%  \begin{equation}
%  \end{equation}
%  \end{linenomath*}


%% Enter Figures and Tables near as possible to where they are first mentioned:
%
% DO NOT USE \psfrag or \subfigure commands.
%
% Figure captions go below the figure.
%\section{= enter section title =}
%Text here ===>>>
%\section{= enter section title =}
%Text here ===>>>
%%

%  Numbered lines in equations:
%  To add line numbers to lines in equations,
%  \begin{linenomath*}
%  \begin{equation}
%  \end{equation}
%  \end{linenomath*}



%% Enter Figures and Tables near as possible to where they are first mentioned:
%
% DO NOT USE \psfrag or \subfigure commands.
%
% Figure captions go below the figure.
% Table titles go above tables;  other caption information
%  should be placed in last line of the table, using
% \multicolumn2l{$^a$ This is a table note.}
%
%----------------
% EXAMPLE FIGURES
%
% \begin{figure}
% \includegraphics{example.png}
% \caption{caption}
% \end{figure}
%
% Giving latex a width will help it to scale the figure properly. A simple trick is to use \textwidth. Try this if large figures run off the side of the page.
% \begin{figure}
% \noindent\includegraphics[width=\textwidth]{anothersample.png}
%\caption{caption}
%\label{pngfiguresample}
%\end{figure}
%
%
% If you get an error about an unknown bounding box, try specifying the width and height of the figure with the natwidth and natheight options. This is common when trying to add a PDF figure without pdflatex.
% \begin{figure}
% \noindent\includegraphics[natwidth=800px,natheight=600px]{samplefigure.pdf}
%\caption{caption}
%\label{pdffiguresample}
%\end{figure}
%
%
% PDFLatex does not seem to be able to process EPS figures. You may want to try the epstopdf package.
%

%
% ---------------
% EXAMPLE TABLE
%
% \begin{table}
% \caption{Time of the Transition Between Phase 1 and Phase 2$^{a}$}
% \centering
% \begin{tabular}{l c}
% \hline
%  Run  & Time (min)  \\
% \hline
%   $l1$  & 260   \\
%   $l2$  & 300   \\
%   $l3$  & 340   \\
%   $h1$  & 270   \\
%   $h2$  & 250   \\
%   $h3$  & 380   \\
%   $r1$  & 370   \\
%   $r2$  & 390   \\
% \hline
% \multicolumn{2}{l}{$^{a}$Footnote text here.}
% \end{tabular}
% \end{table}

%% SIDEWAYS FIGURE and TABLE
% AGU prefers the use of {sidewaystable} over {landscapetable} as it causes fewer problems.
%
% \begin{sidewaysfigure}
% \includegraphics[width=20pc]{figsamp}
% \caption{caption here}
% \label{newfig}
% \end{sidewaysfigure}
%
%  \begin{sidewaystable}
%  \caption{Caption here}
% \label{tab:signif_gap_clos}
%  \begin{tabular}{ccc}
% one&two&three\\
% four&five&six
%  \end{tabular}
%  \end{sidewaystable}

%% If using numbered lines, please surround equations with \begin{linenomath*}...\end{linenomath*}
%\begin{linenomath*}
%\begin{equation}
%y|{f} \sim g(m, \sigma),
%\end{equation}
%\end{linenomath*}

%%% End of body of article

%%%%%%%%%%%%%%%%%%%%%%%%%%%%%%%%
%% Optional Appendix goes here
%
% The \appendix command resets counters and redefines section heads
%
% After typing \appendix
%
%\section{Here Is Appendix Title}
% will show
% A: Here Is Appendix Title
%
%\appendix
%\section{Here is a sample appendix}

%%%%%%%%%%%%%%%%%%%%%%%%%%%%%%%%%%%%%%%%%%%%%%%%%%%%%%%%%%%%%%%%
%
% Optional Glossary, Notation or Acronym section goes here:
%
%%%%%%%%%%%%%%
% Glossary is only allowed in Reviews of Geophysics
%  \begin{glossary}
%  \term{Term}
%   Term Definition here
%  \term{Term}
%   Term Definition here
%  \term{Term}
%   Term Definition here
%  \end{glossary}

%
%%%%%%%%%%%%%%
% Acronyms
%   \begin{acronyms}
%   \acro{Acronym}
%   Definition here
%   \acro{EMOS}
%   Ensemble model output statistics
%   \acro{ECMWF}
%   Centre for Medium-Range Weather Forecasts
%   \end{acronyms}

%
%%%%%%%%%%%%%%
% Notation
%   \begin{notation}
%   \notation{$a+b$} Notation Definition here
%   \notation{$e=mc^2$}
%   Equation in German-born physicist Albert Einstein's theory of special
%  relativity that showed that the increased relativistic mass ($m$) of a
%  body comes from the energy of motion of the body—that is, its kinetic
%  energy ($E$)—divided by the speed of light squared ($c^2$).
%   \end{notation}




%%%%%%%%%%%%%%%%%%%%%%%%%%%%%%%%%%%%%%%%%%%%%%%%%%%%%%%%%%%%%%%%
%
%  ACKNOWLEDGMENTS
%
% The acknowledgments must list:
%
% >>>>	A statement that indicates to the reader where the data
% 	supporting the conclusions can be obtained (for example, in the
% 	references, tables, supporting information, and other databases).
%
% 	All funding sources related to this work from all authors
%
% 	Any real or perceived financial conflicts of interests for any
%	author
%
% 	Other affiliations for any author that may be perceived as
% 	having a conflict of interest with respect to the results of this
% 	paper.
%
%
% It is also the appropriate place to thank colleagues and other contributors.
% AGU does not normally allow dedications.
%% ------------------------------------------------------------------------ %%
%% References and Citations

%%%%%%%%%%%%%%%%%%%%%%%%%%%%%%%%%%%%%%%%%%%%%%%
%
% \bibliography{<name of your .bib file>} don't specify the file extension
%
% don't specify bibliography style
%%%%%%%%%%%%%%%%%%%%%%%%%%%%%%%%%%%%%%%%%%%%%%%
%
%
%Reference citation instructions and examples:
%
% Please use ONLY \cite and \citeA for reference citations.
% \cite for parenthetical references
% ...as shown in recent studies (Simpson et al., 2019)
% \citeA for in-text citations
% ...Simpson et al. (2019) have shown...
%
%
%...as shown by \citeA{jskilby}.
%...as shown by \citeA{lewin76}, \citeA{carson86}, \citeA{bartoldy02}, and \citeA{rinaldi03}.
%...has been shown \cite{jskilbye}.
%...has been shown \cite{lewin76,carson86,bartoldy02,rinaldi03}.
%... \cite <i.e.>[]{lewin76,carson86,bartoldy02,rinaldi03}.
%...has been shown by \cite <e.g.,>[and others]{lewin76}.
%
% apacite uses < > for prenotes and [ ] for postnotes
% DO NOT use other cite commands (e.g., \citet, \citep, \citeyear, \nocite, \citealp, etc.).
%



\end{document}



%%%% More Information and Advice:

%% ------------------------------------------------------------------------ %%
%
%  SECTION HEADS
%
%% ------------------------------------------------------------------------ %%

% Capitalize the first letter of each word (except for
% prepositions, conjunctions, and articles that are
% three or fewer letters).

% AGU follows standard outline style; therefore, there cannot be a section 1 without
% a section 2, or a section 2.3.1 without a section 2.3.2.
% Please make sure your section numbers are balanced.
% ---------------
% Level 1 head
%
% Use the \section{} command to identify level 1 heads;
% type the appropriate head wording between the curly
% brackets, as shown below.
%
%An example:
%\section{Level 1 Head: Introduction}
%
% ---------------
% Level 2 head
%
% Use the \subsection{} command to identify level 2 heads.
%An example:
%\subsection{Level 2 Head}
%
% ---------------
% Level 3 head
%
% Use the \subsubsection{} command to identify level 3 heads
%An example:
%\subsubsection{Level 3 Head}
%
%---------------
% Level 4 head
%
% Use the \subsubsubsection{} command to identify level 3 heads
% An example:
%\subsubsubsection{Level 4 Head} An example.
%
%% ------------------------------------------------------------------------ %%
%
%  IN-TEXT LISTS
%
%% ------------------------------------------------------------------------ %%
%
% Do not use bulleted lists; enumerated lists are okay.
% \begin{enumerate}
% \item
% \item
% \item
% \end{enumerate}
%
%% ------------------------------------------------------------------------ %%
%
%  EQUATIONS
%
%% ------------------------------------------------------------------------ %%

% Single-line equations are centered.
% Equation arrays will appear left-aligned.

Math coded inside display math mode \[ ...\]
 will not be numbered, e.g.,:
 \[ x^2=y^2 + z^2\]

 Math coded inside \begin{equation} and \end{equation} will
 be automatically numbered, e.g.,:
 \begin{equation}
 x^2=y^2 + z^2
 \end{equation}


% To create multiline equations, use the
% \begin{eqnarray} and \end{eqnarray} environment
% as demonstrated below.
\begin{eqnarray}
  x_{1} & = & (x - x_{0}) \cos \Theta \nonumber \\
        && + (y - y_{0}) \sin \Theta  \nonumber \\
  y_{1} & = & -(x - x_{0}) \sin \Theta \nonumber \\
        && + (y - y_{0}) \cos \Theta.
\end{eqnarray}

%If you don't want an equation number, use the star form:
%\begin{eqnarray*}...\end{eqnarray*}

% Break each line at a sign of operation
% (+, -, etc.) if possible, with the sign of operation
% on the new line.

% Indent second and subsequent lines to align with
% the first character following the equal sign on the
% first line.

% Use an \hspace{} command to insert horizontal space
% into your equation if necessary. Place an appropriate
% unit of measure between the curly braces, e.g.
% \hspace{1in}; you may have to experiment to achieve
% the correct amount of space.


%% ------------------------------------------------------------------------ %%
%
%  EQUATION NUMBERING: COUNTER
%
%% ------------------------------------------------------------------------ %%

% You may change equation numbering by resetting
% the equation counter or by explicitly numbering
% an equation.

% To explicitly number an equation, type \eqnum{}
% (with the desired number between the brackets)
% after the \begin{equation} or \begin{eqnarray}
% command.  The \eqnum{} command will affect only
% the equation it appears with; LaTeX will number
% any equations appearing later in the manuscript
% according to the equation counter.
%

% If you have a multiline equation that needs only
% one equation number, use a \nonumber command in
% front of the double backslashes (\\) as shown in
% the multiline equation above.

% If you are using line numbers, remember to surround
% equations with \begin{linenomath*}...\end{linenomath*}

%  To add line numbers to lines in equations:
%  \begin{linenomath*}
%  \begin{equation}
%  \end{equation}
%  \end{linenomath*}



