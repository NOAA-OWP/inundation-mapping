%% 
%% Copyright 2007-2020 Elsevier Ltd
%% 
%% This file is part of the 'Elsarticle Bundle'.
%% ---------------------------------------------
%% 
%% It may be distributed under the conditions of the LaTeX Project Public
%% License, either version 1.2 of this license or (at your option) any
%% later version.  The latest version of this license is in
%%    http://www.latex-project.org/lppl.txt
%% and version 1.2 or later is part of all distributions of LaTeX
%% version 1999/12/01 or later.
%% 
%% The list of all files belonging to the 'Elsarticle Bundle' is
%% given in the file `manifest.txt'.
%% 
%% Template article for Elsevier's document class `elsarticle'
%% with harvard style bibliographic references

\documentclass[preprint,review,12pt]{dependencies/elsarticle}

%% Use the option review to obtain double line spacing
%% \documentclass[preprint,review,12pt]{elsarticle}

%% Use the options 1p,twocolumn; 3p; 3p,twocolumn; 5p; or 5p,twocolumn
%% for a journal layout:
%% \documentclass[final,1p,times]{elsarticle}
%% \documentclass[final,1p,times,twocolumn]{elsarticle}
%% \documentclass[final,3p,times]{elsarticle}
%% \documentclass[final,3p,times,twocolumn]{elsarticle}
%% \documentclass[final,5p,times]{elsarticle}
%% \documentclass[final,5p,times,twocolumn]{elsarticle}

%% For including figures, graphicx.sty has been loaded in
%% elsarticle.cls. If you prefer to use the old commands
%% please give \usepackage{epsfig}

%% The amssymb package provides various useful mathematical symbols
\usepackage{amssymb}
%% The amsthm package provides extended theorem environments
%% \usepackage{amsthm}

%% The lineno packages adds line numbers. Start line numbering with
%% \begin{linenumbers}, end it with \end{linenumbers}. Or switch it on
%% for the whole article with \linenumbers.
\usepackage{lineno}
%
%% Additional packages
\usepackage[printonlyused]{acronym}
%\usepackage{hyperref}
%
%%%%%%%%%%%%%%%%%%%%%%%%%%%%%%%%%%%%%%%%%%%%%%%%%%%%%%%%%%%%%%%%%%%%%%%%%%%%%%%%%%%%%%%%%%%%%%%%%
%%%%%%%%%%%%%%%%%%%%%%%%%%%%%%%%%%%%%%%%%%%%%%%%%%%%%%%%%%%%%%%%%%%%%%%%%%%%%%%%%%%%%%%%%%%%%%%%%
\journal{Advances in Water Resources}
%
\begin{document}
%
%%%%%%%%%%%%%%%%%%%%%%%%%%%%%%%%%%%%%%%%%%%%%%%%%%%%%%%%%%%%%%%%%%%%%%%%%%%%%%%%%%%%%%%%%%%%%%%%%
%%%%%%%%%%%%%%%%%%%%%%%%%%%%%%%%%%%%%%%%%%%%%%%%%%%%%%%%%%%%%%%%%%%%%%%%%%%%%%%%%%%%%%%%%%%%%%%%%
%% FRONT MATTER
%%%%%%%%%%%%%%%%%%%%%%%%%%%%%%%%%%%%%%%%%%%%%%%%%%%%%%%%%%%%%%%%%%%%%%%%%%%%%%%%%%%%%%%%%%%%%%%%%
%%%%%%%%%%%%%%%%%%%%%%%%%%%%%%%%%%%%%%%%%%%%%%%%%%%%%%%%%%%%%%%%%%%%%%%%%%%%%%%%%%%%%%%%%%%%%%%%%
\begin{frontmatter}
%
%% Title, authors and addresses
%
%% use the tnoteref command within \title for footnotes;
%% use the tnotetext command for theassociated footnote;
%% use the fnref command within \author or \address for footnotes;
%% use the fntext command for theassociated footnote;
%% use the corref command within \author for corresponding author footnotes;
%% use the cortext command for theassociated footnote;
%% use the ead command for the email address,
%% and the form \ead[url] for the home page:
%% \title{Title\tnoteref{label1}}
%% \tnotetext[label1]{}
%% \author{Name\corref{cor1}\fnref{label2}}
%% \ead{email address}
%% \ead[url]{home page}
%% \fntext[label2]{}
%% \cortext[cor1]{}
%% \affiliation{organization={},
%%             addressline={},
%%             city={},
%%             postcode={},
%%             state={},
%%             country={}}
%% \fntext[label3]{}

%%%%%%%%%%%%%%%%%%%%%%%%%%%%%%%%%%%%%%%%%%%%%%%%%%%%%%%%%%%%%%%%%%%%%%%%%%%%%%%%%%%%%%%%%%%%%%%%%
%% TITLE 
%%%%%%%%%%%%%%%%%%%%%%%%%%%%%%%%%%%%%%%%%%%%%%%%%%%%%%%%%%%%%%%%%%%%%%%%%%%%%%%%%%%%%%%%%%%%%%%%%

\title{Investigating the Effect of Varying the Spatial Resolutions of LiDAR Derived Elevations on Flood Inundation Maps Produced from Height Above Nearest Drainage}

%%%%%%%%%%%%%%%%%%%%%%%%%%%%%%%%%%%%%%%%%%%%%%%%%%%%%%%%%%%%%%%%%%%%%%%%%%%%%%%%%%%%%%%%%%%%%%%%%
%% AUTHORS
%%%%%%%%%%%%%%%%%%%%%%%%%%%%%%%%%%%%%%%%%%%%%%%%%%%%%%%%%%%%%%%%%%%%%%%%%%%%%%%%%%%%%%%%%%%%%%%%%

\author[lynk,nwc,uf]{Fernando Aristizabal\corref{fa_corr}}
\ead{fernando.aristizabal@noaa.gov}
\ead[url]{https://www.linkedin.com/in/fernando-aristizabal}
%\fntext[fa_corr]{Please direct correspondence to Fernando Aristizabal.}
\cortext[fa_corr]{Please direct correspondence to Fernando Aristizabal.}

\author[nwc]{Fernando Salas}
\author[uh]{Taher Chegini}
\author[usgs]{Gregory Petrochenkov}
\author[uf]{Jasmeet Judge}
%
%%%%%%%%%%%%%%%%%%%%%%%%%%%%%%%%%%%%%%%%%%%%%%%%%%%%%%%%%%%%%%%%%%%%%%%%%%%%%%%%%%%%%%%%%%%%%%%%%
%% AFFILIATIONS
%%%%%%%%%%%%%%%%%%%%%%%%%%%%%%%%%%%%%%%%%%%%%%%%%%%%%%%%%%%%%%%%%%%%%%%%%%%%%%%%%%%%%%%%%%%%%%%%%
%
\affiliation[lynk]{
             organization={Lynker},
             addressline={338 E Market St}, 
             city={Leesburg},
             postcode={20176}, 
             state={VA},
             country={USA}
            }

\affiliation[nwc]{
             organization={National Water Center, Office of Water Prediction, National Oceanic and Atmospheric Administration},
             addressline={205 Hackberry Ln}, 
             city={Tuscaloosa},
             postcode={35401}, 
             state={VA},
             country={USA}
            }
%
\affiliation[uf]{
             organization={Center for Remote Sensing, Agricultural and Biological Engineering, University of Florida},
             addressline={1741 Museum Rd}, 
             city={Gainesville},
             postcode={32603}, 
             state={FL},
             country={USA}
            }
%
\affiliation[uh]{
             organization={Civil and Environmental Engineering, University of Houston},
             addressline={4226 Martin Luther King Boulevard}, 
             city={Houston},
             postcode={77204}, 
             state={TX},
             country={USA}
            }
%
\affiliation[usgs]{
             organization={Hydrologic Applied Innovations Lab, New York Water Science Center, United States Geological Survey},
             addressline={425 Jordan Rd}, 
             city={Troy},
             postcode={12180}, 
             state={NY},
             country={USA}
            }
%
%%%%%%%%%%%%%%%%%%%%%%%%%%%%%%%%%%%%%%%%%%%%%%%%%%%%%%%%%%%%%%%%%%%%%%%%%%%%%%%%%%%%%%%%%%%%%%%%%
%% ABSTRACT
%%%%%%%%%%%%%%%%%%%%%%%%%%%%%%%%%%%%%%%%%%%%%%%%%%%%%%%%%%%%%%%%%%%%%%%%%%%%%%%%%%%%%%%%%%%%%%%%%
%
\begin{abstract}

\end{abstract}
%
%%%%%%%%%%%%%%%%%%%%%%%%%%%%%%%%%%%%%%%%%%%%%%%%%%%%%%%%%%%%%%%%%%%%%%%%%%%%%%%%%%%%%%%%%%%%%%%%%
%% Graphical abstract
%%%%%%%%%%%%%%%%%%%%%%%%%%%%%%%%%%%%%%%%%%%%%%%%%%%%%%%%%%%%%%%%%%%%%%%%%%%%%%%%%%%%%%%%%%%%%%%%%
%
%\begin{graphicalabstract}
%\includegraphics{grabs}
%\end{graphicalabstract}
%
%%%%%%%%%%%%%%%%%%%%%%%%%%%%%%%%%%%%%%%%%%%%%%%%%%%%%%%%%%%%%%%%%%%%%%%%%%%%%%%%%%%%%%%%%%%%%%%%%
%% Research highlights
%%%%%%%%%%%%%%%%%%%%%%%%%%%%%%%%%%%%%%%%%%%%%%%%%%%%%%%%%%%%%%%%%%%%%%%%%%%%%%%%%%%%%%%%%%%%%%%%%
%
\begin{highlights}
\item Continental-scale flood inundation mapping requires simplifying assumptions
\item Utilizing LiDAR derived data from the USGS 3DEP program enhances the skill on flood inundation maps produced from \ac{HAND}.
\item Varying the spatial resolution provides no skill enhancement when evaluated at large scales across three study sites and benchmark datasets.
\end{highlights}
%
%%%%%%%%%%%%%%%%%%%%%%%%%%%%%%%%%%%%%%%%%%%%%%%%%%%%%%%%%%%%%%%%%%%%%%%%%%%%%%%%%%%%%%%%%%%%%%%%%
%% KEYWORDS
%%%%%%%%%%%%%%%%%%%%%%%%%%%%%%%%%%%%%%%%%%%%%%%%%%%%%%%%%%%%%%%%%%%%%%%%%%%%%%%%%%%%%%%%%%%%%%%%%
%
%%%%%%%%%%%%%%%%%%%%%%%%%%%%%%%%%
%% MAX SIX WORDS %%%%%%%%%%%%%%%%
%%%%%%%%%%%%%%%%%%%%%%%%%%%%%%%%%
%
\begin{keyword}
Hydrology \sep
Height Above Nearest Drainage \sep
Flood Inundation Mapping \sep
Digital Elevation Maps \sep
Forecasting
%
%% PACS codes here, in the form: \PACS code \sep code
\PACS 92.40.-t \sep 92.40.Qk \sep *92.40.Q- \sep *92.40.qp \sep 89.60.Gg \sep 92.60.Wc \sep 91.10.Jf \sep 84.40.Xb
%
%% MSC codes here, in the form: \MSC code \sep code
%% or \MSC[2008] code \sep code (2000 is the default)
%
\end{keyword}
%
\end{frontmatter}
%
%%%%%%%%%%%%%%%%%%%%%%%%%%%%%%%%%%%%%%%%%%%%%%%%%%%%%%%%%%%%%%%%%%%%%%%%%%%%%%%%%%%%%%%%%%%%%%%%%
%%%%%%%%%%%%%%%%%%%%%%%%%%%%%%%%%%%%%%%%%%%%%%%%%%%%%%%%%%%%%%%%%%%%%%%%%%%%%%%%%%%%%%%%%%%%%%%%%
%% MAIN TEXT
%%%%%%%%%%%%%%%%%%%%%%%%%%%%%%%%%%%%%%%%%%%%%%%%%%%%%%%%%%%%%%%%%%%%%%%%%%%%%%%%%%%%%%%%%%%%%%%%%
%%%%%%%%%%%%%%%%%%%%%%%%%%%%%%%%%%%%%%%%%%%%%%%%%%%%%%%%%%%%%%%%%%%%%%%%%%%%%%%%%%%%%%%%%%%%%%%%%
%
% prints line numbers
\linenumbers
%
%%%%%%%%%%%%%%%%%%%%%%%%%%%%%%%%%%%%%%%%%%%%%%%%%%%%%%%%%%%%%%%%%%%%%%%%%%%%%%%%%%%%%%%%%%%%%%%%%
%% INTRODUCTION
%%%%%%%%%%%%%%%%%%%%%%%%%%%%%%%%%%%%%%%%%%%%%%%%%%%%%%%%%%%%%%%%%%%%%%%%%%%%%%%%%%%%%%%%%%%%%%%%%
\section{Introduction}
\label{sec:introduction}
%
% Flush acronym usage
\acresetall 
%
%%%% Motivates and introduces flooding %%%%
Floods are among the most frequent, damaging, and deadly of natural disasters \citep{doocy2013human,stromberg2007natural,kahn2005death}. 
The frequency and intensity of flood events as well as the exposure of people and property to them have been increasing in recent times driven by secular changes in climate, infrastructure, and demographics \citep{berz2000flood,mallakpour2015changing,downton2005reanalysis,kunkel1999temporal,pielke2000precipitation,corringham2019effect}. 
Unfortunately, these trends are expected to continue placing additional pressure on hydrological extremes \citep{kahn2005death,tabari2020climate,milly2002increasing,wing2018estimates}.
Floods impact mortality and morbidity through drowning or physical trauma at the individual health scale, while increasing the risk of infectious disease at the public health level \citep{jonkman2005global,beinin2012medical,alajo2006cholera,french1983mortality}.
Flooding disrupts systems providing human needs such as transportation routes, supply chains, water delivery, waste management, communications, and energy grids \citep{wijkman2021natural}.
These impacts disproportionately affect certain demographics such as the socioeconomically-disadvantaged, youth, and elderly who are more likely to live in vulnerable areas with less access to educational resources, \acp{EWS}, and the capacity or resources to evacuate impacted areas \citep{kahn2005death,smiley2022social,stromberg2007natural,jonkman2005global,tellman2020using,tellman2021satellite}.
These inequitable impacts further entrench poverty and inequalities \citep{stallings1988conflict,birkmann2010extreme}.
In political terms, severe disasters, including floods, can reduce social order, strain governance systems, collapse social safety nets, increase the risk of social conflict \citep{drury1998disasters,xu2016natural,zahran2009natural}.
These dire consequences motivate adaption and mitigation efforts such as \acp{EWS}, protective infrastructure (e.g. storage, defenses, drainage, infiltration), public awareness and education, and zoning regulations \citep{tumbare2000mitigating,tauhid2018mitigating,charlesworth201115}.

%%%% Motivates and introduces NWM %%%%
Due to the growing consequences and risks presented by increasing flood impacts, \acp{EWS}, or forecasting systems, can help understand future conditions and provide intelligence to furnish adequate warnings to protect life, prevent damages, and enhance resilience \citep{stromberg2007natural,cools2016lessons,unisdr2015making,baudoin2014early,golnaraghi2012overview,unep2012early,liu2018review}.
The early warning of flood disasters at national scales requires the use of continental-scale, forecast hydrology models and modeling frameworks that span intranational political boundaries.
The applications of these models extend beyond \acp{EWS} to provide historical trends for applications in infrastructure planning, public planning, insurance underwriting, and more.
The \ac{OWP}, an office of the \ac{NOAA} along with partners at the \ac{NCAR}, developed such a continental-scale model known as the \ac{US} \ac{NWM} \citep{salas2018towards,gochis2021wrf,cosgrove2019evolution,cohen2018featured,noaa2016national,water2022nwm}.
The \ac{NWM} is based on a configuration of the \ac{WRF-Hydro} model that accounts for land surface processes as well as overland and channel routing \citep{gochis2021wrf,salas2018towards,cosgrove2019evolution}.
Operationally, the \ac{NWM} produces streamflow analysis and forecasts at multiple time horizons depending on location which include the \ac{CONUS}, \ac{PR}, and \ac{HI} \citep{cosgrove2019evolution,noaa2016national,water2022nwm}.
The \ac{NWM} routes streamflow across the \ac{NWM} \ac{V2.1} stream network, based on the \ac{NHDPlusV2} network, is comprised of more than 5.5 million \acp{km} of lines discretized into more than 2.8 million forecast points \citep{aristizabal2022reducing}.
The \ac{NWM} \ac{V2.1} stream network belongs to the \ac{NWM} ``hydrofabric'' defined as a catalog of geospatial layers relevant to hydrology modeling including stream network lines, catchments, reservoirs, and more \citep{water2022nwm,cosgrove2019evolution}.
While streamflow is an important variable for engineering and scientific applications of fluvial flooding, flood inundation stages, extents and depths are much more tangible variables to the stakeholders flood events directly impact.

%%%% Introduces and motivates HAND %%%%
The \acp{SWE}, a system of two hyperbolic partial differential equations, formally govern the flow of fluvial surface water by conserving both mass (first equation) and momentum (second equation) and can be expressed in both the \ac{1D} (Saint Venant Equations) and \ac{2D} forms.
Solving this system in full \ac{2D} form requires numerical methods that can be very cost prohibitive and numerically unstable in an operational setting across continental-scales at high spatial discretizations (10 \ac{m} or higher).
This use case motivates the implementation of an inundation proxy, also known as a zero-physics or a simplified conceptual model, that is agnostic to the \acp{SWE} while still computing accurate fluvial inundation extents and depths for this problem \citep{teng2015rapid}.
\ac{HAND} detrends elevations within \acp{DEM} to compute drainage potentials by normalizing elevations to the nearest, relevant drainage line instead of datums that represent mean sea level \citep{renno2008hand,nobre2011height,nobre2016hand}.
\ac{HAND} as a terrain index has been used extensively for \ac{FIM} purposes from both modeled or observed stream flows and stages \citep{nobre2016hand,afshari2018comparison,garousi2019terrain,johnson2019integrated,zheng2018geoflood,zheng2018river,zhang2018comparative,teng2015rapid,li2022accounting,li2020evaluation}, as well as for assisting the remote sensing detection of fluvial inundation \citep{aristizabal2020high,shastry2019using,aristizabal2021mapping,huang2017comparison,twele2016sentinel}.
\ac{HAND} operates as an inundation proxy by thresholding the relative elevation (or \ac{HAND}) values with a singular river stage value for each catchment corresponding to the drainage area of a given river reach \citep{nobre2016hand,garousi2019terrain,johnson2019integrated,zheng2018geoflood,teng2015rapid,li2020evaluation,liu2016cybergis,maidment2017conceptual,liu2018cybergis,liu2020height,liu2018review}.
When used to generate inundation extents and depths from streamflow, reach-averaged \ac{SRC} sample geometric variables along an entire reach and normalize using the length of the reach to create stage-discharge relationships \citep{zheng2018river,aristizabal2022reducing,godbout2019error}.
These relationships depend on a friction parameter, Manning's n, and used to convert streamflows to stages for eventual \ac{2D} mapping with \ac{HAND}.
Numerous investigations have validated the use of \ac{HAND} for flood mapping applications as a suitable alternative to more sophisticated physics-based techniques for large scale and high resolution use cases \citep{johnson2019integrated,li2020evaluation,li2022comparative,aristizabal2022reducing,nobre2016hand,godbout2019error,afshari2018comparison,zhang2018comparative,teng2015rapid,teng2017flood,diehl2021improving}.

%%%% Motivates and introduces OWP FIM %%%%
Several prior and active large-scale \ac{HAND} implementations catered to operational \ac{EWS} applications including the \ac{NFIE} \citep{maidment2017conceptual,liu2016cybergis,liu2018cybergis}, GeoFlood \citep{zheng2018geoflood}, and \ac{PyGFT}, i.e. \ac{GIS} \citep{petrochenkov2020pygft,verdin2016software}.
The \ac{NFIE} was a broad, inter-institutional effort to apply HAND to the initial versions of the \ac{NWM} which leveraged 10 \ac{m} (1/3 arc-second) elevation data available at the time \citep{maidment2017conceptual,liu2016cybergis,liu2018cybergis} from the \ac{USGS}'s \ac{NED} \citep{gesch2009national,gesch2014accuracy} which now has been subsumed \citep{arundel2018assimilation} to The National Map program \citep{gesch2009national,archuleta2017national}.
\citet{zheng2018geoflood} applies HAND for operational applications to \ac{LiDAR} derived 1 \ac{m} elevation data with a novel least cost, geodesic based stream delineation method \citep{passalacqua2010geometric,passalacqua2012automatic,zheng2018geoflood}.
For applications with the \ac{NWM}, an advanced version of \ac{HAND} coupled with the use of \acp{SRC}, known as \ac{OWP} \ac{FIM}, converts \ac{NWM} analysis, reanalysis, and forecast streamflows to river stages and fluvial inundation depths and extents on an operational basis to \ac{CONUS} while extending the modeling domain to \ac{PR} and \ac{HI} \citep{aristizabal2022reducing,inundationMapping2022}.
\ac{OWP} \ac{FIM} utilizes some of the latest datasets including the \ac{NHDPlusHR}, \ac{NLD}, and \ac{NWM} \ac{V2.1} hydrofabric to enforce drainage, hydrologically relevant features such as levees, and conflation with the forecast stream network \citep{aristizabal2022reducing,inundationMapping2022}.
Additionally, \ac{OWP} \ac{FIM} advanced a fundamental limitation of \ac{HAND} that limits sourcing fluvial inundation only from the nearest, relevant drainage line \citep{mcgehee2016modified,aristizabal2022reducing,zhang2018comparative,li2022comparative,zheng2018geoflood,zheng2018river,nobre2016hand}.
Stream lines of higher Horton-Strahler stream order that could contribute inundation to a given area have no way of extending beyond catchment lines which creates artificial bottlenecks in inundation extents especially along junctions of high order rivers with their lower flow tributaries \citep{aristizabal2022reducing,mcgehee2016modified}.
To resolve this limitation, \ac{OWP} \ac{FIM} disaggregates the \ac{NWM} \ac{V2.1} stream network into segments of effective unit stream order called \acp{LP} \citep{aristizabal2022reducing}.
In terms of terrain data, \ac{OWP} \ac{FIM} utilizes the 10 \ac{m} \ac{DEM} from the \ac{NHDPlusHR} dataset which is based on the \ac{NED} due to its high fidelity and continental availability \citep{aristizabal2022reducing,moore2019user}.
The previous advances in \ac{OWP} \ac{FIM} stopped short of accounting for new \ac{LiDAR} datasets that are now approaching continental scale availability from the \ac{USGS} \citep{aristizabal2022reducing}.

%%%% 3DEP program introduction with LiDAR intro %%%%
The \ac{USGS}'s \ac{3DEP} extends The National Map Program from the already existing 1/3, 1, and 2 arc-second spatial resolution seamless \acp{DEM} products to include a 1 \ac{m}, \ac{LiDAR} derived \ac{DEM} product as well.
While other sources of observation are available for \ac{DEM} derivation such as \ac{InSAR} in Alaska, we focus here on 1 \ac{m} data derived from \ac{LiDAR} and use that synonymously with the term \ac{3DEP}.

3DEP divided up into hydrography, topographic, 
Base specifications from USGS are \citep{arundel20151} for 1m and the rest are \citep{archuleta2017national}.

Lidar technical details \citep{stoker2015usgs}.

NED evaluation \citep{gesch2014accuracy,dobbs2010evaluation}

Compares Lidar data to NED for flooding using HEC-RAS \citep{wang2005comparison}.

The 1 \ac{m} \ac{3DEP} product is a hydro-flattened, topographic bare-earth raster \ac{DEM} gridded to 1 \ac{km} square shaped tiles with 6 pixels of overlap \citep{arundel20151}.
The 1 \ac{m} \ac{DEM} is derived almost entirely from \ac{LiDAR} except in a few places where photogrammetry falling under the specifications of the \ac{LiDAR} data are used \citep{arundel20151}.
According to specifications, the horizontal accuracy of 1 \ac{m} \ac{3DEP} is expected to be within 1 \ac{m} while the vertical accuracies are within 19.6 \ac{cm} and 30 \ac{cm} at the 95\% confidence interval for non-vegetative and vegetative regions, respectively \citep{arundel20151}.
Non-vegetative vertical accuracies fall within 10 \ac{cm} \ac{RSME} \ac{arundel20151}.

Elevation-derived hydrography specifications \citep{terziotti2020elevation}.

% Specifications
- Lidar base specifications \citep{heidemann2012lidar}
- national map seamless products \citep{archuleta2017national}
- 1m specifications \citep{arundel20151}


%%%% assessments on 3DEP data %%%%
Assessments of 3DEP \citep{stoker2022accuracy,kim2020positional,chirico2020evaluating,callahan2022vertical,scott2022statewide,dobbs2010evaluation}
Assessment of airborne Lidar on bathymetry data quality \citep{hilldale2008assessing}.
Assessment of Lidar \citep{tarolli2014high}

%%%% FIM studies on importance of Lidar and elevation data %%%%
Numerous reports and academic studies discuss the principal effect that topographic data quality has on the quality of \ac{FIM}.
The National Enhanced Elevation Assessment conducted by \citet{dewberry2011final} quantified the estimated annual benefits of high resolution, enhanced elevation data by 27 business sectors and concluded flood risk management to have the highest conservative estimate among the sectors in 2015 \ac{USD}.
Several technical reports document that the quality of the elevation map used is of primal significance to the quality of the flood maps produced \citep{national2007elevation,national2009mapping}.

Make sure to mention geoflood \citep{zheng2018geoflood,zheng2019automatic}.
Make sure to mention the three limitations of elevation data include bathymetry, embankment delineations, and anthropogenic features such as bridges.

DEMs and flooding \citep{casas2006topographic,thomas2016quantifying,savage2016does,passalacqua2010geometric,passalacqua2012automatic,munoth2019effects}

Effects of resolution on hilly areas \citep{dai2019effects}

Changed the horizontal resolution of DEM for HAND \citep{li2022accounting}, showed that DEM resolution was the most important factor influencing FIM extent skill when varied alongside water depth and drainage threshold.

Flood specific assessments for 3DEP \citep{carswell20183d,bales2009sources,gesch2018best,podhoranyi2015inaccuracy,lamichhane2018effect,tsubaki2013uncertainty,dobbs2010evaluation,arrighi2019effects,zazo2015analysis,bhuyian2018accounting,gesch2012elevation}.

\citep{garousi2019terrain} using 3m vs 10m with HAND.

Effects of flat areas on FIM quality and overestimation \citep{garousi2019terrain,godbout2019error,jafarzadegan2017based,papaioannou2017probabilistic}

\citep{lopez2018influence} how DEM resolution affects hydro connectivity and inundation extents.


%%%% Motivation paragraph for this study %%%%
With these latest developments in terrain data availability, we propose investigating the integration of \ac{3DEP} data into \ac{OWP} \ac{FIM} for continental-scale inundation forecasting abilities.
We will investigate utilizing \ac{3DEP} data for \ac{HAND} computation to generate the \ac{FIM} hydrofabric.
Additionally, we investigate the utility of varying spatial resolutions from 1, 3, 5, 10, 15, and 20 \ac{m}.
\ac{HAND} depends on the drainage assumptions which requires \acp{DEM} to undergo a long series of enforcement processes to ensure monotonically decreasing elevations with hydrologically correct flow directions \citep{garousi2019terrain,nobre2011height,nobre2016hand,aristizabal2022reducing}.
The resampling of \acp{DEM} into varying spatial resolutions could interact with these hydro-conditioning operations thus influencing the \ac{FIM} hydrofabric and the resulting quality of the \acp{FIM} produced.
We seek to quantify this relationship and assess how these interact with one another to optimize for the quality of continental-scale \acp{FIM} from the \ac{NWM}.

%%%%%%%%%%%%%%%%%%%%%%%%%%%%%%%%%%%%%%%%%%%%%%%%%%%%%%%%%%%%%%%%%%%%%%%%%%%%%%%%%%%%%%%%%%%%%%%%%
%% MATERIAL AND METHODS
%%%%%%%%%%%%%%%%%%%%%%%%%%%%%%%%%%%%%%%%%%%%%%%%%%%%%%%%%%%%%%%%%%%%%%%%%%%%%%%%%%%%%%%%%%%%%%%%%
\section{Material and Methods}
\label{sec:material_and_matheds}
%
\ac{HAND} assumes monotonically decreasing elevations with hydrologically relevant flow paths ensuring all cells in a region drain \citep{renno2008hand,nobre2011height,nobre2016hand}.
This requires the use of extensive hydro-conditioning techniques on \acp{DEM} to enforce drainage and agreement with existing hydrography \citep{aristizabal2022reducing,maidment2017conceptual,liu2016cybergis,liu2020height}.

%%%%%%%%%%%%%%%%%%%%%%%%%%%%%%%%%%%%%%%%%%%%%%%%%%%%%%%%%%%%%%%%
\subsection{\ac{3DEP}}
\label{ssec:3dep}
%
%% make a table of different 3DEP data sources
\begin{table}[h!]
 \begin{center}
  \caption{\acf{3DEP} product specifications.}
  \label{tab:table1}
  \begin{tabular}{c|c|c|c|c|c} % <-- Alignments: 1st column left, 2nd middle and 3rd right, with vertical lines in between
   Program & Product & Arc-Second Resolution & \ac{m} Resolution & Horizontal Accuracy (\ac{m}) & Vertical Accuracy (\ac{m})\\
   \hline
   1 & 1110.1 & a &  &  &  \\
   \hline
   2 & 10.1 & b &  &  &  \\
   \hline
   3 & 23.113231 & c &  &  &  \\
   \hline
  \end{tabular}
 \end{center}
\end{table}

%%%%%%%%%%%%%%%%%%%%%%%%%%%%%%%%%%%%%%%%%%%%%%%%%%%%%%%%%%%%%%%%%%%%%%%%%%%%%%%%%%%%%%%%%%%%%%%%%
%% RESULTS
%%%%%%%%%%%%%%%%%%%%%%%%%%%%%%%%%%%%%%%%%%%%%%%%%%%%%%%%%%%%%%%%%%%%%%%%%%%%%%%%%%%%%%%%%%%%%%%%%
\section{Results}
\label{sec:results}
%

%%%%%%%%%%%%%%%%%%%%%%%%%%%%%%%%%%%%%%%%%%%%%%%%%%%%%%%%%%%%%%%%%%%%%%%%%%%%%%%%%%%%%%%%%%%%%%%%%
%% DISCUSSION
%%%%%%%%%%%%%%%%%%%%%%%%%%%%%%%%%%%%%%%%%%%%%%%%%%%%%%%%%%%%%%%%%%%%%%%%%%%%%%%%%%%%%%%%%%%%%%%%%
\section{Discussion}
\label{sec:discussion}
%

%%%%%%%%%%%%%%%%%%%%%%%%%%%%%%%%%%%%%%%%%%%%%%%%%%%%%%%%%%%%%%%%%%%%%%%%%%%%%%%%%%%%%%%%%%%%%%%%%
%% CONCLUSIONS
%%%%%%%%%%%%%%%%%%%%%%%%%%%%%%%%%%%%%%%%%%%%%%%%%%%%%%%%%%%%%%%%%%%%%%%%%%%%%%%%%%%%%%%%%%%%%%%%%
\section{Conclusions}
\label{sec:conclusions}
%

%%%%%%%%%%%%%%%%%%%%%%%%%%%%%%%%%%%%%%%%%%%%%%%%%%%%%%%%%%%%%%%%%%%%%%%%%%%%%%%%%%%%%%%%%%%%%%%%%
%%%%%%%%%%%%%%%%%%%%%%%%%%%%%%%%%%%%%%%%%%%%%%%%%%%%%%%%%%%%%%%%%%%%%%%%%%%%%%%%%%%%%%%%%%%%%%%%%
% Flush acronym usage
\acresetall 
%
%%%%%%%%%%%%%%%%%%%%%%%%%%%%%%%%%%%%%%%%%%%%%%%%%%%%%%%%%%%%%%%%%%%%%%%%%%%%%%%%%%%%%%%%%%%%%%%%%
%% ACKNOWLEDGMENTS
%%%%%%%%%%%%%%%%%%%%%%%%%%%%%%%%%%%%%%%%%%%%%%%%%%%%%%%%%%%%%%%%%%%%%%%%%%%%%%%%%%%%%%%%%%%%%%%%%
\section{Acknowledgments}
\label{sec:acknowledgments}
%
Collate acknowledgements in a separate section at the end of the article before the references and do not, therefore, include them on the title page, as a footnote to the title or otherwise.
List here those individuals who provided help during the research (e.g., providing language help, writing assistance or proof reading the article, etc.).
%
%%%%%%%%%%%%%%%%%%%%%%%%%%%%%%%%%%%%%%%%%%%%%%%%%%%%%%%%%%%%%%%%%%%%%%%%%%%%%%%%%%%%%%%%%%%%%%%%%
%% FUNDING SOURCES
%%%%%%%%%%%%%%%%%%%%%%%%%%%%%%%%%%%%%%%%%%%%%%%%%%%%%%%%%%%%%%%%%%%%%%%%%%%%%%%%%%%%%%%%%%%%%%%%%
\section{Funding Sources}
\label{sec:funding_sources}
%
List funding sources in this standard way to facilitate compliance to funder's requirements:

Funding: This work was supported by the National Institutes of Health [grant numbers xxxx, yyyy]; the Bill \& Melinda Gates Foundation, Seattle, WA [grant number zzzz]; and the United States Institutes of Peace [grant number aaaa].

It is not necessary to include detailed descriptions on the program or type of grants and awards.
When funding is from a block grant or other resources available to a university, college, or other research institution, submit the name of the institute or organization that provided the funding.
If no funding has been provided for the research, it is recommended to include the following sentence:
This research did not receive any specific grant from funding agencies in the public, commercial, or not-for-profit sectors.
%
%%%%%%%%%%%%%%%%%%%%%%%%%%%%%%%%%%%%%%%%%%%%%%%%%%%%%%%%%%%%%%%%%%%%%%%%%%%%%%%%%%%%%%%%%%%%%%%%%
%%%%%%%%%%%%%%%%%%%%%%%%%%%%%%%%%%%%%%%%%%%%%%%%%%%%%%%%%%%%%%%%%%%%%%%%%%%%%%%%%%%%%%%%%%%%%%%%%
%% APPENDIX
%%%%%%%%%%%%%%%%%%%%%%%%%%%%%%%%%%%%%%%%%%%%%%%%%%%%%%%%%%%%%%%%%%%%%%%%%%%%%%%%%%%%%%%%%%%%%%%%%
%%%%%%%%%%%%%%%%%%%%%%%%%%%%%%%%%%%%%%%%%%%%%%%%%%%%%%%%%%%%%%%%%%%%%%%%%%%%%%%%%%%%%%%%%%%%%%%%%
%
%% The Appendices part is started with the command \appendix;
%% appendix sections are then done as normal sections
\appendix


%% \section{}
%% \label{}
%
%%%%%%%%%%%%%%%%%%%%%%%%%%%%%%%%%%%%%%%%%%%%%%%%%%%%%%%%%%%%%%%%%%%%%%%%%%%%%%%%%%%%%%%%%%%%%%%%%
% ACRONYMS
%%%%%%%%%%%%%%%%%%%%%%%%%%%%%%%%%%%%%%%%%%%%%%%%%%%%%%%%%%%%%%%%%%%%%%%%%%%%%%%%%%%%%%%%%%%%%%%%%
\section{Acronyms}
\label{sec:acronyms}
%
\begin{acronym}
\acro{OWP}{Office of Water Prediction}
\acro{EWS}{early warning system}
\acro{NWM}{National Water Model}
\acro{NOAA}{National Oceanic and Atmospheric Administration}
\acro{NWC}{National Water Center}
\acro{NFIE}{National Flood Interoperability Experiment}
\acro{NWS}{National Water Service}
\acro{NCAR}{National Center for Atmospheric Research}
\acro{WRF-Hydro}{Weather Research and Forecasting Hydro}
\acro{US}{United States}
\acro{USGS}[US Geological Survey]{United States Geological Survey}
\acro{USD}{US Dollar}
\acro{BLE}{Base Level Engineering}
\acro{FEMA}{Federal Emergency Management Agency}
\acro{HAND}{Height Above Nearest Drainage}
\acro{TP}{true positive}
\acro{FP}{false positive}
\acro{TN}{true negative}
\acro{FN}{false negative}
\acro{CSI}{critical success index}
\acro{POD}{probability of detection}
\acro{FAR}{false alarm rate}
\acro{3D}{3-Dimensional}
\acro{2D}{2-Dimensional}
\acro{1D}{1-Dimensional}
\acro{3DEP}{3-Dimensional Elevation Program}
\acro{DEM}{digital elevation model}
\acro{NLD}{National Levee Database}
\acro{NHD}{National Hydrography Dataset}
\acro{NED}{National Elevation Dataset}
\acro{NHDPlus}{National Hydrography Dataset Plus}
\acro{NHDPlusV2}{National Hydrography Dataset Plus Version 2}
\acro{NHDPlusHR}{National Hydrography Dataset Plus High Resolution}
\acro{LP}{level path}
\acro{SRC}{synthetic rating curve}
\acro{SWE}{Shallow Water Equations}
\acro{CONUS}{Continental United States}
\acro{PR}{Puerto Rico}
\acro{HI}{Hawaii}
\acro{V2.1}{Version 2.1}
\acro{V4}{Version 4}
\acro{km}{kilometer}
\acro{cm}{centimeter}
\acro{RMSE}{root mean squared error}
\acro{$km^2$}{square kilometer}
\acro{$m^2$}{square meter}
\acro{m}{meter}
\acro{FIM}{flood inundation map}
\acro{LiDAR}{Light Detection and Ranging}
\acro{PyGFT}{Python GIS Flood Tool}
\acro{GIS}{Geographic Information Systems}
\acro{InSAR}{interferometric synthetic aperture radar}
\end{acronym}
%
%%%%%%%%%%%%%%%%%%%%%%%%%%%%%%%%%%%%%%%%%%%%%%%%%%%%%%%%%%%%%%%%%%%%%%%%%%%%%%%%%%%%%%%%%%%%%%%%%
%% BIBLIOGRAPHY
%%%%%%%%%%%%%%%%%%%%%%%%%%%%%%%%%%%%%%%%%%%%%%%%%%%%%%%%%%%%%%%%%%%%%%%%%%%%%%%%%%%%%%%%%%%%%%%%%
%
%% For citations use: 
%%       \citet{<label>} ==> Jones et al. [21]
%%       \citep{<label>} ==> [21]
%%

%% If you have bibdatabase file and want bibtex to generate the
%% bibitems, please use
%%
\bibliographystyle{elsarticle-num-names} 
\bibliography{bibliography/owp_3dep_2022}

%%%%%%%%%%%%%%%%%%%%%%%%%%%%%%%%%%%%%%%%%%%%%%%%%%%%%%%%%%%%%%%%%%%%%%%%%%%%%%%%%%%%%%%%%%%%%%%%%
%% END
%%%%%%%%%%%%%%%%%%%%%%%%%%%%%%%%%%%%%%%%%%%%%%%%%%%%%%%%%%%%%%%%%%%%%%%%%%%%%%%%%%%%%%%%%%%%%%%%%
\end{document}
\endinput
