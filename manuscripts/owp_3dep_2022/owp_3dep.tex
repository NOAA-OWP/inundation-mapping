%% 
%% Copyright 2007-2020 Elsevier Ltd
%% 
%% This file is part of the 'Elsarticle Bundle'.
%% ---------------------------------------------
%% 
%% It may be distributed under the conditions of the LaTeX Project Public
%% License, either version 1.2 of this license or (at your option) any
%% later version.  The latest version of this license is in
%%    http://www.latex-project.org/lppl.txt
%% and version 1.2 or later is part of all distributions of LaTeX
%% version 1999/12/01 or later.
%% 
%% The list of all files belonging to the 'Elsarticle Bundle' is
%% given in the file `manifest.txt'.
%% 
%% Template article for Elsevier's document class `elsarticle'
%% with harvard style bibliographic references

\documentclass[preprint,12pt]{dependencies/elsarticle}

%% Use the option review to obtain double line spacing
%% \documentclass[preprint,review,12pt]{elsarticle}

%% Use the options 1p,twocolumn; 3p; 3p,twocolumn; 5p; or 5p,twocolumn
%% for a journal layout:
%% \documentclass[final,1p,times]{elsarticle}
%% \documentclass[final,1p,times,twocolumn]{elsarticle}
%% \documentclass[final,3p,times]{elsarticle}
%% \documentclass[final,3p,times,twocolumn]{elsarticle}
%% \documentclass[final,5p,times]{elsarticle}
%% \documentclass[final,5p,times,twocolumn]{elsarticle}

%% For including figures, graphicx.sty has been loaded in
%% elsarticle.cls. If you prefer to use the old commands
%% please give \usepackage{epsfig}

%% The amssymb package provides various useful mathematical symbols
\usepackage{amssymb}
%% The amsthm package provides extended theorem environments
%% \usepackage{amsthm}

%% The lineno packages adds line numbers. Start line numbering with
%% \begin{linenumbers}, end it with \end{linenumbers}. Or switch it on
%% for the whole article with \linenumbers.
\usepackage{lineno}

%%%%%%%%%%%%%%%%%%%%%%%%%%%%%%%%%%%%%%%%%%%%%%%%%%%%%%%%%%%%%
%% Additional packages
\usepackage{acronym}

%%%%%%%%%%%%%%%%%%%%%%%%%%%%%%%%%%%%%%%%%%%%%%%%%%%%%%%%%%%%%

\journal{Advances in Water Resources}

\begin{document}

\begin{frontmatter}

%% Title, authors and addresses

%% use the tnoteref command within \title for footnotes;
%% use the tnotetext command for theassociated footnote;
%% use the fnref command within \author or \address for footnotes;
%% use the fntext command for theassociated footnote;
%% use the corref command within \author for corresponding author footnotes;
%% use the cortext command for theassociated footnote;
%% use the ead command for the email address,
%% and the form \ead[url] for the home page:
%% \title{Title\tnoteref{label1}}
%% \tnotetext[label1]{}
%% \author{Name\corref{cor1}\fnref{label2}}
%% \ead{email address}
%% \ead[url]{home page}
%% \fntext[label2]{}
%% \cortext[cor1]{}
%% \affiliation{organization={},
%%             addressline={},
%%             city={},
%%             postcode={},
%%             state={},
%%             country={}}
%% \fntext[label3]{}

\title{Varying the Spatial Resolution of LiDAR Derived Elevations to Balance Skill and Computational Costs for Flood Inundation Mapping Applications}

%% use optional labels to link authors explicitly to addresses:
%% \author[label1,label2]{}
%% \affiliation[label1]{organization={},
%%             addressline={},
%%             city={},
%%             postcode={},
%%             state={},
%%             country={}}
%%
%% \affiliation[label2]{organization={},
%%             addressline={},
%%             city={},
%%             postcode={},
%%             state={},
%%             country={}}

\author[lynk,nwc,uf]{Fernando Aristizabal}
\author[nwc]{Fernando Salas}
\author[uh]{Taher Chegini}
\author[usgs]{Gregory Petrochenkov}
\author[uf]{Jasmeet Judge}

\affiliation[lynk]{
             organization={Lynker}, %Department and Organization
             addressline={338 E Market St}, 
             city={Leesburg},
             postcode={20176}, 
             state={VA},
             country={USA}
            }

\affiliation[nwc]{
             organization={National Water Center, Office of Water Prediction, National Oceanic and Atmospheric Administration}, %Department and Organization
             addressline={205 Hackberry Ln}, 
             city={Tuscaloosa},
             postcode={35401}, 
             state={VA},
             country={USA}
            }

\affiliation[uf]{
             organization={Center for Remote Sensing, Agricultural and Biological Engineering, University of Florida}, %Department and Organization
             addressline={1741 Museum Rd}, 
             city={Gainesville},
             postcode={32603}, 
             state={FL},
             country={USA}
            }

\affiliation[uh]{
             organization={Civil and Environmental Engineering, University of Houston},
             addressline={4226 Martin Luther King Boulevard}, 
             city={Houston},
             postcode={77204}, 
             state={TX},
             country={USA}
            }

\affiliation[usgs]{
             organization={Hydrologic Applied Innovations Lab, New York Water Science Center, United States Geological Survey},
             addressline={425 Jordan Rd}, 
             city={Troy},
             postcode={12180}, 
             state={NY},
             country={USA}
            }


\begin{abstract}
%% Text of abstract

\end{abstract}

%%Graphical abstract
%\begin{graphicalabstract}
%\includegraphics{grabs}
%\end{graphicalabstract}

%%Research highlights
\begin{highlights}
\item Continental-scale flood inundation mapping requires simplifying assumptions
\item Utilizing LiDAR derived data from the USGS 3DEP program enhances the skill on flood inundation maps produced from HAND.
\item Varying the spatial resolution provides no skill enhancement when evaluated at large scales across three study sites and benchmark datasets.
\end{highlights}

\begin{keyword}
%% keywords here, in the form: keyword \sep keyword

%% PACS codes here, in the form: \PACS code \sep code

%% MSC codes here, in the form: \MSC code \sep code
%% or \MSC[2008] code \sep code (2000 is the default)

\end{keyword}

\end{frontmatter}

%%%%%%%%%%%%%%%%%%%%%%%%%%%%%%%%%%%%%%%%%%%%%%%%%%%%%%%%%%%%%%%%
%%%%%%%%%%%%%%%%%%%%%%%%%%%%%%%%%%%%%%%%%%%%%%%%%%%%%%%%%%%%%%%%
% Acronyms
%%%%%%%%%%%%%%%%%%%%%%%%%%%%%%%%%%%%%%%%%%%%%%%%%%%%%%%%%%%%%%%%
%%%%%%%%%%%%%%%%%%%%%%%%%%%%%%%%%%%%%%%%%%%%%%%%%%%%%%%%%%%%%%%%
\begin{acronym}

\acro{OWP}[OWP]{Office of Water Prediction}

\end{acronym}

\linenumbers


%%%%%%%%%%%%%%%%%%%%%%%%%%%%%%%%%%%%%%%%%%%%%%%%%%%%%%%%%%%%%%%%
%%%%%%%%%%%%%%%%%%%%%%%%%%%%%%%%%%%%%%%%%%%%%%%%%%%%%%%%%%%%%%%%
%% main text
%%%%%%%%%%%%%%%%%%%%%%%%%%%%%%%%%%%%%%%%%%%%%%%%%%%%%%%%%%%%%%%%
%%%%%%%%%%%%%%%%%%%%%%%%%%%%%%%%%%%%%%%%%%%%%%%%%%%%%%%%%%%%%%%%

%%%%%%%%%%%%%%%%%%%%%%%%%%%%%%%%%%%%%%%%%%%%%%%%%%%%%%%%%%%%%%%%
%%%%%%%%%%%%%%%%%%%%%%%%%%%%%%%%%%%%%%%%%%%%%%%%%%%%%%%%%%%%%%%%
\section{Introduction}
\label{sec:introduction}
%
Floods are among the most frequent, damaging, and deadly of natural disasters \citet{doocy2013human,stromberg2007natural,kahn2005death}. 
What's worse, trends in flood events have been increasing in frequency and intensity driven by climate change, anthropogenic development, \citet{berz2000flood,mallakpour2015changing,downton2005reanalysis,kunkel1999temporal,pielke2000precipitation,corringham2019effect} with more hydrologic extremes expected in the future \citet{kahn2005death,tabari2020climate,milly2002increasing,wing2018estimates}.
Floods impact mortality and morbidity through drowning, trauma, or increased risk of infectious disease \citet{jonkman2005global,beinin2012medical,alajo2006cholera,french1983mortality}.
Flooding impacts population inequitably \citet{kahn2005death,smiley2021social,stromberg2007natural,jonkman2005global,tellman2020using,tellman2021satellite}.
Early-warning systems can help \citet{stromberg2007natural}
The \ac{OWP}.

%% The Appendices part is started with the command \appendix;
%% appendix sections are then done as normal sections
%% \appendix

%% \section{}
%% \label{}

%% For citations use: 
%%       \citet{<label>} ==> Jones et al. [21]
%%       \citep{<label>} ==> [21]
%%

%% If you have bibdatabase file and want bibtex to generate the
%% bibitems, please use
%%
\bibliographystyle{elsarticle-num-names} 
\bibliography{bibliography/owp_3dep_2022}

%% else use the following coding to input the bibitems directly in the
%% TeX file.

%\begin{thebibliography}{00}

%% \bibitem[Author(year)]{label}
%% Text of bibliographic item

%\bibitem[ ()]{}

%\end{thebibliography}


\end{document}
\endinput
%%
%% End of file `elsarticle-template-num-names.tex'.
