%  LaTeX support: latex@mdpi.com 
%  For support, please attach all files needed for compiling as well as the log file, and specify your operating system, LaTeX version, and LaTeX editor.

%=================================================================
\documentclass[water,article,submit,pdftex,moreauthors]{dependencies/Definitions/mdpi} 
% For posting an early version of this manuscript as a preprint, you may use "preprints" as the journal and change "submit" to "accept". The document class line would be, e.g., \documentclass[preprints,article,accept,moreauthors,pdftex]{mdpi}. This is especially recommended for submission to arXiv, where line numbers should be removed before posting. For preprints.org, the editorial staff will make this change immediately prior to posting.

%--------------------
% Class Options:
%--------------------
%----------
% journal
%----------
% Choose between the following MDPI journals:
% acoustics, actuators, addictions, admsci, adolescents, aerospace, agriculture, agriengineering, agronomy, ai, algorithms, allergies, alloys, analytica, animals, antibiotics, antibodies, antioxidants, applbiosci, appliedchem, appliedmath, applmech, applmicrobiol, applnano, applsci, aquacj, architecture, arts, asc, asi, astronomy, atmosphere, atoms, audiolres, automation, axioms, bacteria, batteries, bdcc, behavsci, beverages, biochem, bioengineering, biologics, biology, biomass, biomechanics, biomed, biomedicines, biomedinformatics, biomimetics, biomolecules, biophysica, biosensors, biotech, birds, bloods, blsf, brainsci, breath, buildings, businesses, cancers, carbon, cardiogenetics, catalysts, cells, ceramics, challenges, chemengineering, chemistry, chemosensors, chemproc, children, chips, cimb, civileng, cleantechnol, climate, clinpract, clockssleep, cmd, coasts, coatings, colloids, colorants, commodities, compounds, computation, computers, condensedmatter, conservation, constrmater, cosmetics, covid, crops, cryptography, crystals, csmf, ctn, curroncol, currophthalmol, cyber, dairy, data, dentistry, dermato, dermatopathology, designs, diabetology, diagnostics, dietetics, digital, disabilities, diseases, diversity, dna, drones, dynamics, earth, ebj, ecologies, econometrics, economies, education, ejihpe, electricity, electrochem, electronicmat, electronics, encyclopedia, endocrines, energies, eng, engproc, ent, entomology, entropy, environments, environsciproc, epidemiologia, epigenomes, est, fermentation, fibers, fintech, fire, fishes, fluids, foods, forecasting, forensicsci, forests, foundations, fractalfract, fuels, futureinternet, futureparasites, futurepharmacol, futurephys, futuretransp, galaxies, games, gases, gastroent, gastrointestdisord, gels, genealogy, genes, geographies, geohazards, geomatics, geosciences, geotechnics, geriatrics, hazardousmatters, healthcare, hearts, hemato, heritage, highthroughput, histories, horticulturae, humanities, humans, hydrobiology, hydrogen, hydrology, hygiene, idr, ijerph, ijfs, ijgi, ijms, ijns, ijtm, ijtpp, immuno, informatics, information, infrastructures, inorganics, insects, instruments, inventions, iot, j, jal, jcdd, jcm, jcp, jcs, jdb, jeta, jfb, jfmk, jimaging, jintelligence, jlpea, jmmp, jmp, jmse, jne, jnt, jof, joitmc, jor, journalmedia, jox, jpm, jrfm, jsan, jtaer, jzbg, kidney, kidneydial, knowledge, land, languages, laws, life, liquids, literature, livers, logics, logistics, lubricants, lymphatics, machines, macromol, magnetism, magnetochemistry, make, marinedrugs, materials, materproc, mathematics, mca, measurements, medicina, medicines, medsci, membranes, merits, metabolites, metals, meteorology, methane, metrology, micro, microarrays, microbiolres, micromachines, microorganisms, microplastics, minerals, mining, modelling, molbank, molecules, mps, msf, mti, muscles, nanoenergyadv, nanomanufacturing, nanomaterials, ncrna, network, neuroglia, neurolint, neurosci, nitrogen, notspecified, nri, nursrep, nutraceuticals, nutrients, obesities, oceans, ohbm, onco, oncopathology, optics, oral, organics, organoids, osteology, oxygen, parasites, parasitologia, particles, pathogens, pathophysiology, pediatrrep, pharmaceuticals, pharmaceutics, pharmacoepidemiology, pharmacy, philosophies, photochem, photonics, phycology, physchem, physics, physiologia, plants, plasma, pollutants, polymers, polysaccharides, poultry, powders, preprints, proceedings, processes, prosthesis, proteomes, psf, psych, psychiatryint, psychoactives, publications, quantumrep, quaternary, qubs, radiation, reactions, recycling, regeneration, religions, remotesensing, reports, reprodmed, resources, rheumato, risks, robotics, ruminants, safety, sci, scipharm, seeds, sensors, separations, sexes, signals, sinusitis, skins, smartcities, sna, societies, socsci, software, soilsystems, solar, solids, sports, standards, stats, stresses, surfaces, surgeries, suschem, sustainability, symmetry, synbio, systems, taxonomy, technologies, telecom, test, textiles, thalassrep, thermo, tomography, tourismhosp, toxics, toxins, transplantology, transportation, traumacare, traumas, tropicalmed, universe, urbansci, uro, vaccines, vehicles, venereology, vetsci, vibration, viruses, vision, waste, water, wem, wevj, wind, women, world, youth, zoonoticdis 

%---------
% article
%---------
% The default type of manuscript is "article", but can be replaced by: 
% abstract, addendum, article, book, bookreview, briefreport, casereport, comment, commentary, communication, conferenceproceedings, correction, conferencereport, entry, expressionofconcern, extendedabstract, datadescriptor, editorial, essay, erratum, hypothesis, interestingimage, obituary, opinion, projectreport, reply, retraction, review, perspective, protocol, shortnote, studyprotocol, systematicreview, supfile, technicalnote, viewpoint, guidelines, registeredreport, tutorial
% supfile = supplementary materials

%----------
% submit
%----------
% The class option "submit" will be changed to "accept" by the Editorial Office when the paper is accepted. This will only make changes to the frontpage (e.g., the logo of the journal will get visible), the headings, and the copyright information. Also, line numbering will be removed. Journal info and pagination for accepted papers will also be assigned by the Editorial Office.

%------------------
% moreauthors
%------------------
% If there is only one author the class option oneauthor should be used. Otherwise use the class option moreauthors.

%---------
% pdftex
%---------
% The option pdftex is for use with pdfLaTeX. If eps figures are used, remove the option pdftex and use LaTeX and dvi2pdf.

%=================================================================
% MDPI internal commands
\firstpage{1} 
\makeatletter 
\setcounter{page}{\@firstpage} 
\makeatother
\pubvolume{1}
\issuenum{1}
\articlenumber{0}
\pubyear{2022}
\copyrightyear{2022}
%\externaleditor{Academic Editor: Firstname Lastname}
\datereceived{} 
%\daterevised{} % Only for the journal Acoustics
\dateaccepted{} 
\datepublished{} 
%\datecorrected{} % Corrected papers include a "Corrected: XXX" date in the original paper.
%\dateretracted{} % Corrected papers include a "Retracted: XXX" date in the original paper.
\hreflink{https://doi.org/} % If needed use \linebreak
%\doinum{}
%------------------------------------------------------------------
% The following line should be uncommented if the LaTeX file is uploaded to arXiv.org
%\pdfoutput=1

%=================================================================
% Add packages and commands here. The following packages are loaded in our class file: fontenc, inputenc, calc, indentfirst, fancyhdr, graphicx, epstopdf, lastpage, ifthen, lineno, float, amsmath, setspace, enumitem, mathpazo, booktabs, titlesec, etoolbox, tabto, xcolor, soul, multirow, microtype, tikz, totcount, changepage, attrib, upgreek, cleveref, amsthm, hyphenat, natbib, hyperref, footmisc, url, geometry, newfloat, caption
%
\usepackage[printonlyused]{acronym}
\usepackage{multirow}
\usepackage{comment}
\usepackage{url}
\usepackage{amssymb}
%
%=================================================================
%% Please use the following mathematics environments: Theorem, Lemma, Corollary, Proposition, Characterization, Property, Problem, Example, ExamplesandDefinitions, Hypothesis, Remark, Definition, Notation, Assumption
%% For proofs, please use the proof environment (the amsthm package is loaded by the MDPI class).

%=================================================================
% Full title of the paper (Capitalized)
\Title{On the Effects of Varying Spatial Resolutions of 3DEP Data on Flood Inundation Extents Produced from Height Above Nearest Drainage}

% MDPI internal command: Title for citation in the left column
\TitleCitation{Title}

% Author Orchid ID: enter ID or remove command
\newcommand{\orcidauthorA}{0000-0003-2525-4712} % Add \orcidA{} behind the author's name
\newcommand{\orcidauthorB}{0000-0002-5430-6000} % Add \orcidB{} behind the author's name
\newcommand{\orcidauthorC}{0000-0003-2939-4114} % Add \orcidC{} behind the author's name
\newcommand{\orcidauthorD}{0000-0001-9247-821X} % Add \orcidD{} behind the author's name
\newcommand{\orcidauthorE}{0000-0001-9849-7411} % Add \orcidE{} behind the author's name

% Authors, for the paper (add full first names)
\Author{Fernando Aristizabal $^{1,2,3,}$\orcidA{}*, Taher Chegini $^{4,}$\orcidB, Fernando Salas $^{2,}$\orcidC, Gregory Petrochenkov $^{5,}$\orcidD and Jasmeet Judge $^{3,}$\orcidE}

%\longauthorlist{yes}

% MDPI internal command: Authors, for metadata in PDF
\AuthorNames{Fernando Aristizabal, Taher Chegini, Gregory Petrochenkov, Fernando Salas, and Jasmeet Judge}

% MDPI internal command: Authors, for citation in the left column
\AuthorCitation{Aristizabal, F.; Chegini, T.; Petrochenkov, G.; Salas, F.; Judge, J.}
% If this is a Chicago style journal: Lastname, Firstname, Firstname Lastname, and Firstname Lastname.

% Affiliations / Addresses (Add [1] after \address if there is only one affiliation.)
\address{%
$^{1}$ \quad Lynker, 338 E Market St, Leesburg, VA, 20176, USA  \\
$^{2}$ \quad National Water Center, Office of Water Prediction, National Oceanic and Atmospheric Administration, 205 Hackberry Ln, Tuscaloosa, AL, 35401, USA \\
$^{3}$ \quad Center for Remote Sensing, Agricultural and Biological Engineering, University of Florida, 1741 Museum Rd, Gainesville, FL, 32603, USA \\
$^{4}$ \quad Civil and Environmental Engineering, University of Houston, 4226 Martin Luther King Boulevard, Houston, TX, 77204, USA \\
$^{5}$ \quad Hydrologic Applied Innovations Lab, New York Water Science Center, United States Geological Survey, 425 Jordan Rd, Troy, NY, 12180, USA}

% Contact information of the corresponding author
\corres{Correspondence: fernando.aristizabal@noaa.gov}

% Current address and/or shared authorship
%\firstnote{Current address: Affiliation 3.} 
%\secondnote{These authors contributed equally to this work.}
% The commands \thirdnote{} till \eighthnote{} are available for further notes

%\simplesumm{} % Simple summary

%\conference{} % An extended version of a conference paper

% Abstract (Do not insert blank lines, i.e. \\) 
\abstract{A single paragraph of about 200 words maximum. For research articles, abstracts should give a pertinent overview of the work. We strongly encourage authors to use the following style of structured abstracts, but without headings: (1) Background: place the question addressed in a broad context and highlight the purpose of the study; (2) Methods: describe briefly the main methods or treatments applied; (3) Results: summarize the article's main findings; (4) Conclusions: indicate the main conclusions or interpretations. The abstract should be an objective representation of the article, it must not contain results which are not presented and substantiated in the main text and should not exaggerate the main conclusions.}

% Keywords
\keyword{keyword 1; keyword 2; keyword 3 (List three to ten pertinent keywords specific to the article; yet reasonably common within the subject discipline.)} 

% The fields PACS, MSC, and JEL may be left empty or commented out if not applicable
\PACS{92.40.Qk}  % surface water, water resources, water supply, floods, lakes, rivers, streamflow,

%%%%%%%%%%%%%%%%%%%%%%%%%%%%%%%%%%%%%%%%%%
\begin{document}

%%%%%%%%%%%%%%%%%%%%%%%%%%%%%%%%%%%%%%%%%%
\begin{comment}
\setcounter{section}{-1} %% Remove this when starting to work on the template.
\section{How to Use this Template}

The template details the sections that can be used in a manuscript. Note that the order and names of article sections may differ from the requirements of the journal (e.g., the positioning of the Materials and Methods section). Please check the instructions on the authors' page of the journal to verify the correct order and names. For any questions, please contact the editorial office of the journal or support@mdpi.com. For LaTeX-related questions please contact latex@mdpi.com.%\endnote{This is an endnote.} % To use endnotes, please un-comment \printendnotes below (before References). Only journal Laws uses \footnote.

% The order of the section titles is different for some journals. Please refer to the "Instructions for Authors” on the journal homepage.
\end{comment}

%%%%%%%%%%%%%%%%%%%%%%%%%%%%%%%%%%%%%%%%%%
\section{Introduction}
\label{sec:introduction}
%
% Flush acronym usage
\acresetall 
%
%%%% Motivates and introduces flooding %%%%
Floods are among the most frequent, damaging, and deadly of natural disasters \cite{doocy2013human,stromberg2007natural,kahn2005death}. 
The frequency and intensity of flood events as well as the exposure of people and property to them have been increasing in recent times driven by secular changes in climate, infrastructure, and demographics \cite{berz2000flood,mallakpour2015changing,downton2005reanalysis,kunkel1999temporal,pielke2000precipitation,corringham2019effect}. 
Unfortunately, these trends are expected to continue placing additional pressure on hydrological extremes \cite{kahn2005death,tabari2020climate,milly2002increasing,wing2018estimates}.
Floods impact mortality and morbidity through drowning or physical trauma at the individual health scale, while increasing the risk of infectious disease at the public health level \cite{jonkman2005global,beinin2012medical,alajo2006cholera,french1983mortality}.
Flooding disrupts systems providing human needs such as transportation routes, supply chains, water delivery, waste management, communications, and energy grids \cite{wijkman2021natural}.
These impacts disproportionately affect certain demographics such as the socioeconomically-disadvantaged, youth, and elderly who are more likely to live in vulnerable areas with less access to educational resources, \acp{EWS}, and the capacity or resources to evacuate impacted areas \cite{kahn2005death,smiley2022social,stromberg2007natural,jonkman2005global,tellman2020using,tellman2021satellite}.
These inequitable impacts further entrench poverty and inequalities \cite{stallings1988conflict,birkmann2010extreme}.
In political terms, severe disasters, including floods, can reduce social order, strain governance systems, collapse social safety nets, increase the risk of social conflict \cite{drury1998disasters,xu2016natural,zahran2009natural}.
These dire consequences motivate adaption and mitigation efforts such as \acp{EWS}, protective infrastructure (e.g. storage, defenses, drainage, infiltration), public awareness and education, and zoning regulations \cite{tumbare2000mitigating,tauhid2018mitigating,charlesworth201115}.

%%%% Motivates and introduces NWM %%%%
Due to the growing consequences and risks presented by increasing flood impacts, \acp{EWS}, or forecasting systems, can help understand future conditions and provide intelligence to furnish adequate warnings to protect life, prevent damages, and enhance resilience \cite{stromberg2007natural,cools2016lessons,unisdr2015making,baudoin2014early,golnaraghi2012overview,unep2012early,liu2018review}.
The early warning of flood disasters at national scales requires the use of continental-scale, forecast hydrology models and modeling frameworks that span intranational political boundaries.
The applications of these models extend beyond \acp{EWS} to provide historical trends for applications in infrastructure planning, public planning, insurance underwriting, and more.
The \ac{OWP}, an office of the \ac{NOAA} along with partners at the \ac{NCAR}, developed such a continental-scale model known as the \ac{US} \ac{NWM} \cite{salas2018towards,gochis2021wrf,cosgrove2019evolution,cohen2018featured,noaa2016national,water2022nwm}.
The \ac{NWM} is based on a configuration of the \ac{WRF-Hydro} model that accounts for land surface processes as well as overland and channel routing \cite{gochis2021wrf,salas2018towards,cosgrove2019evolution}.
Operationally, the \ac{NWM} produces streamflow analysis and forecasts at multiple time horizons depending on location which include the \ac{CONUS}, \ac{PR}, and \ac{HI} \cite{cosgrove2019evolution,noaa2016national,water2022nwm}.
The \ac{NWM} routes streamflow across the \ac{NWM} \ac{V2.1} stream network, based on the \ac{NHDPlusV2} network, is comprised of more than 5.5 million \acp{km} of lines discretized into more than 2.8 million forecast points \cite{aristizabal2022extending}.
The \ac{NWM} \ac{V2.1} stream network belongs to the \ac{NWM} ``hydrofabric'' defined as a catalog of geospatial layers relevant to hydrology modeling including stream network lines, catchments, reservoirs, and more \cite{water2022nwm,cosgrove2019evolution}.
While streamflow is an important variable for engineering and scientific applications of fluvial flooding, flood inundation stages, extents and depths are much more tangible variables to the stakeholders flood events directly impact.

%%%% Introduces and motivates HAND %%%%
The \ac{SWE}, a system of two hyperbolic partial differential equations, formally govern the flow of fluvial surface water by conserving both mass (first equation) and momentum (second equation) and can be expressed in both the \ac{1D} (Saint Venant Equations) and \ac{2D} forms.
Solving this system in full \ac{2D} form requires numerical methods that can be very cost prohibitive and numerically unstable in an operational setting across continental-scales at high spatial discretizations (10 \ac{m} or higher).
This use case motivates the implementation of an inundation proxy, also known as a zero-physics or a simplified conceptual model, that is agnostic to the \acp{SWE} while still computing accurate fluvial inundation extents and depths for this problem \cite{teng2015rapid,bates2000simple}.
\ac{HAND} detrends elevations within \acp{DEM} to compute drainage potentials by normalizing elevations to the nearest, relevant drainage line instead of datums that represent mean sea level \cite{renno2008hand,nobre2011height,nobre2016hand}.
\ac{HAND} as a terrain index has been used extensively for \ac{FIM} purposes from both modeled or observed stream flows and stages \cite{nobre2016hand,afshari2018comparison,garousi2019terrain,johnson2019integrated,zheng2018geoflood,zheng2018river,zhang2018comparative,teng2015rapid,li2022accounting,li2020evaluation}, as well as for assisting the remote sensing detection of fluvial inundation \cite{aristizabal2020high,shastry2019using,aristizabal2021mapping,huang2017comparison,twele2016sentinel}.
\ac{HAND} operates as an inundation proxy by thresholding the relative elevation (or \ac{HAND}) values with a singular river stage value for each catchment corresponding to the drainage area of a given river reach \cite{nobre2016hand,garousi2019terrain,johnson2019integrated,zheng2018geoflood,teng2015rapid,li2020evaluation,liu2016cybergis,maidment2017conceptual,liu2018cybergis,liu2020height,liu2018review}.
When used to generate inundation extents and depths from streamflow, reach-averaged \ac{SRC} sample geometric variables along an entire reach and normalize using the length of the reach to create stage-discharge relationships \cite{zheng2018river,aristizabal2022extending,godbout2019error}.
These relationships depend on a friction parameter, Manning's n, and used to convert streamflows to stages for eventual \ac{2D} mapping with \ac{HAND}.
Numerous investigations have validated the use of \ac{HAND} for flood mapping applications as a suitable alternative to more sophisticated physics-based techniques for large scale and high resolution use cases \cite{johnson2019integrated,li2020evaluation,li2022comparative,aristizabal2022extending,nobre2016hand,godbout2019error,afshari2018comparison,zhang2018comparative,teng2015rapid,teng2017flood,diehl2021improving,hocini2021performance,bates2003optimal}.

%%%% Motivates and introduces OWP FIM %%%%
Several prior and active large-scale \ac{HAND} implementations catered to operational \ac{EWS} applications including the \ac{NFIE} \cite{maidment2017conceptual,liu2016cybergis,liu2018cybergis}, GeoFlood \cite{zheng2018geoflood,hocini2021performance,d2022identification,carruthers2021assessment,zheng2022application}, and \ac{PyGFT}, i.e. \ac{GIS} \cite{petrochenkov2020pygft,verdin2016software}.
The \ac{NFIE} was a broad, inter-institutional, and pioneering effort to apply HAND to the initial versions of the \ac{NWM} which leveraged 1/3 arc-second (10 \ac{m}) seamless elevation data available at the time \cite{maidment2017conceptual,liu2016cybergis,liu2018cybergis} from the \ac{USGS}'s \ac{NED} \cite{gesch2002national,gesch2007digital}.
\citet{zheng2018geoflood} applied HAND for operational applications with 1/27 arc-second (1 \ac{m}) elevation data with a novel least cost, geodesic based stream delineation method \cite{passalacqua2010geometric,passalacqua2012automatic,zheng2018geoflood,zheng2019automatic,carruthers2021assessment,d2022identification,zheng2022application}.
For applications with the \ac{NWM}, an advanced version of \ac{HAND} coupled with the use of \acp{SRC}, known as \ac{OWP} \ac{FIM}, converts \ac{NWM} analysis, reanalysis, and forecast streamflows to river stages and fluvial inundation depths and extents on an operational basis to \ac{CONUS} while extending the modeling domain to \ac{PR} and \ac{HI} \cite{aristizabal2022extending,inundationMapping2022}.
\ac{OWP} \ac{FIM} utilizes some of the latest datasets including the \ac{NHDPlusHR} \cite{moore2019user}, \ac{NLD} \cite{engineers2016national}, and the \ac{NWM} \ac{V2.1} hydrofabric \cite{water2022nwm,noaa2016national,nwm2022hydrofabric,gochis2021wrf}.
These datasets enforce hydrologically relevant features such as levees and the general location of stream lines to facilitate conflation with the forecast stream network \cite{aristizabal2022extending,inundationMapping2022}.
Additionally, \ac{OWP} \ac{FIM} advanced a fundamental limitation of \ac{HAND} that limits sourcing fluvial inundation only from the nearest, relevant drainage line \cite{mcgehee2016modified,aristizabal2022extending,zhang2018comparative,li2022comparative,zheng2018geoflood,zheng2018river,nobre2016hand}.
Stream lines of higher Horton-Strahler stream order that could contribute inundation to a given area have no way of extending beyond catchment lines which creates artificial bottlenecks in inundation extents especially along junctions of high order rivers with their lower flow tributaries \cite{aristizabal2022extending,mcgehee2016modified}.
To resolve this limitation, \ac{OWP} \ac{FIM} disaggregates the \ac{NWM} \ac{V2.1} stream network into segments of effective unit stream order called \acp{LP} \cite{aristizabal2022extending}.
In terms of terrain data, \ac{OWP} \ac{FIM} utilizes the 10 \ac{m} \ac{DEM} from the \ac{NHDPlusHR} elevation dataset which is the elevation basis, derived in batches from \ac{3DEP}, for additional hydrography products within the \ac{NHDPlusHR} \cite{aristizabal2022extending,moore2019user}.
The previous advances in \ac{OWP} \ac{FIM} stopped short of accounting for \ac{LiDAR} point elevation observations \cite{aristizabal2022extending} that are now nearing their first collection cycle to form a novel seamless, continental scale \ac{DEM} from \ac{3DEP} \cite{usgs2022status,usgs2022partnerships}.

%%%% 3DEP program introduction %%%%
Broad scale terrain information in the form of \acp{DEM} is fundamental to all \ac{FIM} models and a significant influence on skill \cite{bales2009sources,dobbs2010evaluation,wang2005comparison,merwade2008uncertainty,witt2015evaluation}. 
The \ac{NGP}, under the \ac{USGS}, is the primary authority on collecting, processing, and maintaining terrestrial elevation data within the \ac{US} in collaboration with Federal partners within the \ac{NDEP} \cite{omb2016circularA16,dewberry2011final,national2007elevation,national2009mapping,sugarbaker20143d}.
The \ac{NED} \cite{gesch2002national,gesch2007digital}, forms the seamless elevation layers of the \ac{TNM} \cite{gesch2009national,archuleta2017national,arundel2015preparing,arundel2018assimilation,kelmelis2003national}.
Prior to the introduction of \ac{3DEP}, \ac{TNM} was originally composed of three seamless \acp{DEM} at 1/3 (10 \ac{m}), 1 (30 \ac{m}), and 2 (90 \ac{m}) arc-second resolutions produced from a variety legacy sources including digital photogrammetry, cartographic contours, mapped hydrography, and elevations from \ac{SRTM} . \cite{gesch2002national,gesch2007digital,arundel2015preparing}.
High quality elevations derived from \ac{LiDAR} and \ac{InSAR} have been integrated into \ac{TNM} seamless elevation products as made available prior to and after the introduction of \ac{3DEP} \cite{snyder2013national,gesch2002national,arundel2015preparing}.
Work by \citet{gesch2014accuracy},\cite{gesch2007digital}, and \citet{dobbs2010evaluation} illustrated that the inclusion of higher quality elevation data sources had significant improvement in the accuracy of \ac{NED} data when compared to the \ac{NGS} \cite{roman2010geodesy}.
\citet{gesch2014accuracy} identified that the \ac{NED} 1/3 arc-second \ac{DEM}, as of April 2013, had a mean error of -0.29 \ac{m} with an \ac{RMSE} of 1.55 \ac{m} when compared to over 25 thousand reference points. 
At the time of evaluation, the \ac{NED} was subject to legacy, lower quality data sources dating almost a century in the past \cite{sugarbaker20143d,gesch2014accuracy,gesch2007digital}.
This reduction in error and its impact on people and commerce \cite{dewberry2011final} motivated action on collection of elevation data from higher quality data sources \cite{sugarbaker20143d}. 

%%% 3DEP AND LIDAR %%%%%
%%% NEED LIDAR REFERENCES %%%%
\ac{3DEP} is a national, multi-organizational effort by the \ac{NDEP} to survey elevations with high quality sensors in response to growing stakeholder needs on a recurring collection cycle of no greater than 8 \acp{year} \cite{dewberry2011final,snyder2013national,sugarbaker20143d}.
\ac{3DEP} leverages two main collection technologies including \ac{LiDAR} for the \ac{CONUS}, \ac{HI}, and \ac{US} territories as well as \ac{InSAR} for \ac{AK}.
\ac{LiDAR}, the collection source of focus in this study, is a light reflection technology that beams concentrated powerful light and receives the return while recording the travel time and intensity of return. 
\ac{LiDAR} sensors are mounted on top of a variety of mobile or static platforms whose positions are geo-tracked as they collect \ac{LiDAR} returns.
The travel time of the returns, along with knowledge of the speed of light, serve as a relative positioning of the target(s) referenced to a common vertical datum while the intensities serve as indicators of what the target(s) represent.
Modes within the relationship of return intensities with respect to travel time/distance from the \ac{LiDAR} wave forms can be indicative of vegetation or other \acp{LULC} that reflect signals at varying distances and magnitudes and influence elevation errors \cite{gesch2014accuracy}.
These modes can be discretized into varying \ac{DEM} products representing bare earth, structures, or canopy elevations.
The horizontal and vertical accuracies and the horizontal resolutions of terrain observations derived from \ac{LiDAR}, and even the consequential economic benefits \cite{dewberry2011final,dewberry2022nation}, are dependent on a variety of sensor, platform, target, and collection specifications and practices such as nominal pulse spacing, nominal pulse density, and \ac{LULC} of the target \cite{heidemann2018lidar,passalacqua2015analyzing,smith2019determining,salach2018accuracy,gesch2014accuracy}.
\ac{LiDAR} produces point cloud datasets which are scattered, geo-referenced points representing full wave forms or discretized return intensities.
Various assessments of the vertical accuracries of \ac{LiDAR} point clouds have yielded satisfactory results in agreement with \ac{3DEP} requirements \cite{stoker2022accuracy,kim2022absolute,callahan2022vertical,kim2022absolute,salach2018accuracy,passalacqua2015analyzing}.
Point clouds must undergo a series of operations to produce analysis ready, seamless \acp{DEM} \cite{passalacqua2015analyzing}.

%%%% 3DEP DEMs and Accuracies %%%%
\ac{3DEP} extends \ac{TNM} to include a 1/27 arc-second (1 \ac{m}), \ac{LiDAR} derived \ac{DEM} product for \ac{CONUS}, \ac{HI}, and \ac{US} territories as well as a 1/2 arc-second (5 \ac{m}) \ac{DEM} for \ac{AK} derived from \ac{InSAR} \cite{sugarbaker20143d,stoker2015usgs}.
To create bare earth \acp{DEM}, \ac{LiDAR} observations must undergo a series of processes that filter out returns from vegetation, anthropogenic, and other features then grid the observations with resampling methods \cite{passalacqua2015analyzing}.
The 1 \ac{m} \ac{3DEP} product is a \ac{hydro-flattened}, topographic, and bare-earth raster \ac{DEM} gridded to 1 \ac{km} square shaped tiles with 6 pixels of overlap \cite{arundel20151}.
Hydro-flattening refers to a process in which hydrologic features such as lakes, reservoirs, streams, rivers, and more are flattened in elevation for bathymetric regions from lower bank to lower bank represented by breaklines \cite{archuleta2017national,maune2018digital}.
This flattening excludes along gradient directions, parallel to the direction of the breaklines, for hydrologic features that naturally exhibit water conveyance such as streams, rivers, and long reservoirs \cite{arundel20151}.
This process includes elevations underneath bridges that are not accurately observed from topographic \ac{LiDAR}.
According to specifications, the horizontal accuracy of 1 \ac{m} \ac{3DEP} is expected to be within 1 \ac{m} while the vertical accuracies are within 19.6 \ac{cm} and 30 \ac{cm} at the 95\% confidence interval for non-vegetative and vegetative regions, respectively \cite{arundel20151,heidemann2018lidar}.
Non-vegetative vertical accuracies fall within 10 \ac{cm} \ac{RMSE} \cite{arundel20151,heidemann2018lidar}.
 Work by \citet{stoker2022accuracy,callahan2022vertical,kim2022absolute} have verified the vertical accuracies and general quality of the \acp{DEM} for \ac{3DEP} specifications.

%%%% FIM studies on importance of Lidar and elevation data %%%%
As previously stated, the features of topographic information including its source, accuracies, resolutions, and processing methods are of primary importance to the quality of \acp{FIM} \cite{national2007elevation,national2009mapping,carswell20183d,bales2009sources}.
Uncertainties in elevations and the processes used to determine them are propogated into \ac{FIM} extents.
Uncertainty in FIM \cite{merwade2008uncertainty} including data sources, gridding, resampling, resolution, mosaicing, hydrologic conditioning, and more. 


\cite{tarolli2014high} good review of channel extraction and other issues surrounding elevations and resolutions. 

Flood specific assessments for 3DEP \cite{carswell20183d,bales2009sources,gesch2018best,podhoranyi2015inaccuracy,lamichhane2018effect,tsubaki2013uncertainty,dobbs2010evaluation,arrighi2019effects,zazo2015analysis,bhuyian2018accounting,gesch2012elevation,witt2015evaluation}.

Make sure to mention the three limitations of elevation data include bathymetry, embankment delineations, and anthropogenic features such as bridges.

DEMs and flooding \cite{casas2006topographic,thomas2016quantifying,savage2016does,passalacqua2010geometric,passalacqua2012automatic,munoth2019effects}
 - Compares Lidar data to NED for flooding using HEC-RAS \cite{wang2005comparison}.

Effects of resolution on hilly areas \cite{dai2019effects}

Concluded 10m DEM was good enough for hydro applications \cite{zhang1994digital}

more FIM and elevations \cite{werner2001impact,omer2003impact,bates2003optimal,tate2002creating,colby2000modeling}.

Changed the horizontal resolution of DEM for HAND \cite{li2022accounting}, showed that DEM resolution was the most important factor influencing FIM extent skill when varied alongside water depth and drainage threshold.
\cite{garousi2019terrain} using 3m vs 10m with HAND.

Effects of flat areas on FIM quality and overestimation \cite{garousi2019terrain,godbout2019error,jafarzadegan2017based,papaioannou2017probabilistic}

\cite{lopez2018influence} how DEM resolution affects hydro connectivity and inundation extents.


%%%% Motivation paragraph for this study %%%%
Previous efforts with \ac{OWP} \ac{FIM} avoided use of 1 \ac{m} \ac{3DEP} \ac{DEM} products due to lack of spatial coverage and no seamless data availability.
As the 1 \ac{m} \ac{3DEP} product rapidly approaches full \ac{CONUS} coverage in 2023, we propose investigating the integration of \ac{3DEP} data into \ac{OWP} \ac{FIM} for continental-scale inundation forecasting abilities \cite{usgs2022status,usgs2022partnerships}.
We will investigate utilizing \ac{3DEP} data for \ac{HAND} computation to generate the \ac{FIM} hydrofabric.
Additionally, we investigate the utility of varying spatial resolutions from 1, 3, 5, 10, 15, and 20 \ac{m}.
\ac{HAND} depends on the drainage assumptions which requires \acp{DEM} to undergo a long series of enforcement processes to ensure monotonically decreasing elevations with hydrologically correct flow directions \cite{garousi2019terrain,nobre2011height,nobre2016hand,aristizabal2022extending}.
The resampling of \acp{DEM} into varying spatial resolutions could interact with these \ac{hydro-conditioning} operations thus influencing the \ac{FIM} hydrofabric and the resulting quality of the \acp{FIM} produced.
As validation, we utilize \ac{1D} \ac{HEC-RAS} modeled flood inundation extents from both the \ac{BLE} published by \ac{InFRM} and from a novel package that automates the generation of \ac{FIM} from existing \ac{1D} \ac{HEC-RAS} models.
As a third source of validation, we utilize proprietary \ac{FIM} derived from remote sensing earth observation \ac{SAR} furnished by ICEYE.
By varying the spatial resolution of \ac{3DEP} \acp{DEM}, we seek to understand the relationship between spatial resolution and \ac{FIM} skill produced from \ac{HAND} that requires significant \ac{DEM} manipulations to satisfy inherent assumptions. 

%%%%%%%%%%%%%%%%%%%%%%%%%%%%%%%%%%%%%%%%%%
\section{Materials and Methods}
\label{sec:material_and_matheds}

Materials and Methods should be described with sufficient details to allow others to replicate and build on published results. Please note that publication of your manuscript implicates that you must make all materials, data, computer code, and protocols associated with the publication available to readers. Please disclose at the submission stage any restrictions on the availability of materials or information. New methods and protocols should be described in detail while well-established methods can be briefly described and appropriately cited.

Research manuscripts reporting large datasets that are deposited in a publicly avail-able database should specify where the data have been deposited and provide the relevant accession numbers. If the accession numbers have not yet been obtained at the time of submission, please state that they will be provided during review. They must be provided prior to publication.

Interventionary studies involving animals or humans, and other studies require ethical approval must list the authority that provided approval and the corresponding ethical approval code.
\begin{quote}
This is an example of a quote.
\end{quote}

\begin{comment}
%%%%%%%%%%%%%%%%%%%%%%%%%%%%%%%%%%%%%%%%%%
\section{Results}

This section may be divided by subheadings. It should provide a concise and precise description of the experimental results, their interpretation as well as the experimental conclusions that can be drawn.
\subsection{Subsection}
\subsubsection{Subsubsection}

Bulleted lists look like this:
\begin{itemize}
\item	First bullet;
\item	Second bullet;
\item	Third bullet.
\end{itemize}

Numbered lists can be added as follows:
\begin{enumerate}
\item	First item; 
\item	Second item;
\item	Third item.
\end{enumerate}

The text continues here. 

\subsection{Figures, Tables and Schemes}

All figures and tables should be cited in the main text as Figure~\ref{fig1}, Table~\ref{tab1}, Table~\ref{tab2}, etc.

\begin{figure}[H]
\includegraphics[width=10.5 cm]{dependencies/Definitions/logo-mdpi}
\caption{This is a figure. Schemes follow the same formatting. If there are multiple panels, they should be listed as: (\textbf{a}) Description of what is contained in the first panel. (\textbf{b}) Description of what is contained in the second panel. Figures should be placed in the main text near to the first time they are cited. A caption on a single line should be centered.\label{fig1}}
\end{figure}   
\unskip

\begin{table}[H] 
\caption{This is a table caption. Tables should be placed in the main text near to the first time they are~cited.\label{tab1}}
\newcolumntype{C}{>{\centering\arraybackslash}X}
\begin{tabularx}{\textwidth}{CCC}
\toprule
\textbf{Title 1}	& \textbf{Title 2}	& \textbf{Title 3}\\
\midrule
Entry 1		& Data			& Data\\
Entry 2		& Data			& Data\\
\bottomrule
\end{tabularx}
\end{table}
\unskip

\begin{table}[H]
\caption{This is a wide table.\label{tab2}}
	\begin{adjustwidth}{-\extralength}{0cm}
		\newcolumntype{C}{>{\centering\arraybackslash}X}
		\begin{tabularx}{\fulllength}{CCCC}
			\toprule
			\textbf{Title 1}	& \textbf{Title 2}	& \textbf{Title 3}     & \textbf{Title 4}\\
			\midrule
			Entry 1		& Data			& Data			& Data\\
			Entry 2		& Data			& Data			& Data \textsuperscript{1}\\
			\bottomrule
		\end{tabularx}
	\end{adjustwidth}
	\noindent{\footnotesize{\textsuperscript{1} This is a table footnote.}}
\end{table}

%\begin{listing}[H]
%\caption{Title of the listing}
%\rule{\columnwidth}{1pt}
%\raggedright Text of the listing. In font size footnotesize, small, or normalsize. Preferred format: left aligned and single spaced. Preferred border format: top border line and bottom border line.
%\rule{\columnwidth}{1pt}
%\end{listing}

Text.

Text.

\subsection{Formatting of Mathematical Components}

This is the example 1 of equation:
\begin{linenomath}
\begin{equation}
a = 1,
\end{equation}
\end{linenomath}
the text following an equation need not be a new paragraph. Please punctuate equations as regular text.
%% If the documentclass option "submit" is chosen, please insert a blank line before and after any math environment (equation and eqnarray environments). This ensures correct linenumbering. The blank line should be removed when the documentclass option is changed to "accept" because the text following an equation should not be a new paragraph.

This is the example 2 of equation:
\begin{adjustwidth}{-\extralength}{0cm}
\begin{equation}
a = b + c + d + e + f + g + h + i + j + k + l + m + n + o + p + q + r + s + t + u + v + w + x + y + z
\end{equation}
\end{adjustwidth}

% Example of a page in landscape format (with table and table footnote).
%\startlandscape
%\begin{table}[H] %% Table in wide page
%\caption{This is a very wide table.\label{tab3}}
%	\begin{tabularx}{\textwidth}{CCCC}
%		\toprule
%		\textbf{Title 1}	& \textbf{Title 2}	& \textbf{Title 3}	& \textbf{Title 4}\\
%		\midrule
%		Entry 1		& Data			& Data			& This cell has some longer content that runs over two lines.\\
%		Entry 2		& Data			& Data			& Data\textsuperscript{1}\\
%		\bottomrule
%	\end{tabularx}
%	\begin{adjustwidth}{+\extralength}{0cm}
%		\noindent\footnotesize{\textsuperscript{1} This is a table footnote.}
%	\end{adjustwidth}
%\end{table}
%\finishlandscape

% Example of a figure that spans the whole page width. The same concept works for tables, too.
\begin{figure}[H]
\begin{adjustwidth}{-\extralength}{0cm}
\centering
\includegraphics[width=13.5cm]{dependencies/Definitions/logo-mdpi}
\end{adjustwidth}
\caption{This is a wide figure.\label{fig2}}
\end{figure}  

Please punctuate equations as regular text. Theorem-type environments (including propositions, lemmas, corollaries etc.) can be formatted as follows:
%% Example of a theorem:
\begin{Theorem}
Example text of a theorem.
\end{Theorem}

The text continues here. Proofs must be formatted as follows:

%% Example of a proof:
\begin{proof}[Proof of Theorem 1]
Text of the proof. Note that the phrase ``of Theorem 1'' is optional if it is clear which theorem is being referred to.
\end{proof}
The text continues here.

%%%%%%%%%%%%%%%%%%%%%%%%%%%%%%%%%%%%%%%%%%
\section{Discussion}

Authors should discuss the results and how they can be interpreted from the perspective of previous studies and of the working hypotheses. The findings and their implications should be discussed in the broadest context possible. Future research directions may also be highlighted.

%%%%%%%%%%%%%%%%%%%%%%%%%%%%%%%%%%%%%%%%%%
\section{Conclusions}

% Flush acronym usage
\acresetall 

This section is not mandatory, but can be added to the manuscript if the discussion is unusually long or complex.

%%%%%%%%%%%%%%%%%%%%%%%%%%%%%%%%%%%%%%%%%%
\section{Patents}

This section is not mandatory, but may be added if there are patents resulting from the work reported in this manuscript.

%%%%%%%%%%%%%%%%%%%%%%%%%%%%%%%%%%%%%%%%%%
\vspace{6pt} 

%%%%%%%%%%%%%%%%%%%%%%%%%%%%%%%%%%%%%%%%%%
%% optional
%\supplementary{The following supporting information can be downloaded at:  \linksupplementary{s1}, Figure S1: title; Table S1: title; Video S1: title.}

% Only for the journal Methods and Protocols:
% If you wish to submit a video article, please do so with any other supplementary material.
% \supplementary{The following supporting information can be downloaded at: \linksupplementary{s1}, Figure S1: title; Table S1: title; Video S1: title. A supporting video article is available at doi: link.}

%%%%%%%%%%%%%%%%%%%%%%%%%%%%%%%%%%%%%%%%%%
\authorcontributions{For research articles with several authors, a short paragraph specifying their individual contributions must be provided. The following statements should be used ``Conceptualization, X.X. and Y.Y.; methodology, X.X.; software, X.X.; validation, X.X., Y.Y. and Z.Z.; formal analysis, X.X.; investigation, X.X.; resources, X.X.; data curation, X.X.; writing---original draft preparation, X.X.; writing---review and editing, X.X.; visualization, X.X.; supervision, X.X.; project administration, X.X.; funding acquisition, Y.Y. All authors have read and agreed to the published version of the manuscript.'', please turn to the  \href{http://img.mdpi.org/data/contributor-role-instruction.pdf}{CRediT taxonomy} for the term explanation. Authorship must be limited to those who have contributed substantially to the work~reported.}

\funding{Please add: ``This research received no external funding'' or ``This research was funded by NAME OF FUNDER grant number XXX.'' and  and ``The APC was funded by XXX''. Check carefully that the details given are accurate and use the standard spelling of funding agency names at \url{https://search.crossref.org/funding}, any errors may affect your future funding.}

\institutionalreview{In this section, you should add the Institutional Review Board Statement and approval number, if relevant to your study. You might choose to exclude this statement if the study did not require ethical approval. Please note that the Editorial Office might ask you for further information. Please add “The study was conducted in accordance with the Declaration of Helsinki, and approved by the Institutional Review Board (or Ethics Committee) of NAME OF INSTITUTE (protocol code XXX and date of approval).” for studies involving humans. OR “The animal study protocol was approved by the Institutional Review Board (or Ethics Committee) of NAME OF INSTITUTE (protocol code XXX and date of approval).” for studies involving animals. OR “Ethical review and approval were waived for this study due to REASON (please provide a detailed justification).” OR “Not applicable” for studies not involving humans or animals.}

\informedconsent{Any research article describing a study involving humans should contain this statement. Please add ``Informed consent was obtained from all subjects involved in the study.'' OR ``Patient consent was waived due to REASON (please provide a detailed justification).'' OR ``Not applicable'' for studies not involving humans. You might also choose to exclude this statement if the study did not involve humans.

Written informed consent for publication must be obtained from participating patients who can be identified (including by the patients themselves). Please state ``Written informed consent has been obtained from the patient(s) to publish this paper'' if applicable.}

\dataavailability{In this section, please provide details regarding where data supporting reported results can be found, including links to publicly archived datasets analyzed or generated during the study. Please refer to suggested Data Availability Statements in section ``MDPI Research Data Policies'' at \url{https://www.mdpi.com/ethics}. If the study did not report any data, you might add ``Not applicable'' here.} 

\acknowledgments{In this section you can acknowledge any support given which is not covered by the author contribution or funding sections. This may include administrative and technical support, or donations in kind (e.g., materials used for experiments).}

\conflictsofinterest{Declare conflicts of interest or state ``The authors declare no conflict of interest.'' Authors must identify and declare any personal circumstances or interest that may be perceived as inappropriately influencing the representation or interpretation of reported research results. Any role of the funders in the design of the study; in the collection, analyses or interpretation of data; in the writing of the manuscript; or in the decision to publish the results must be declared in this section. If there is no role, please state ``The funders had no role in the design of the study; in the collection, analyses, or interpretation of data; in the writing of the manuscript; or in the decision to publish the~results''.} 

%%%%%%%%%%%%%%%%%%%%%%%%%%%%%%%%%%%%%%%%%%
%% Optional
\sampleavailability{Samples of the compounds ... are available from the authors.}

%% Only for journal Encyclopedia
%\entrylink{The Link to this entry published on the encyclopedia platform.}

\abbreviations{Abbreviations}{
The following abbreviations are used in this manuscript:\\

\noindent 
\begin{tabular}{@{}ll}
MDPI & Multidisciplinary Digital Publishing Institute\\
DOAJ & Directory of open access journals\\
TLA & Three letter acronym\\
LD & Linear dichroism
\end{tabular}
}

%%%%%%%%%%%%%%%%%%%%%%%%%%%%%%%%%%%%%%%%%%
%% Optional
\appendixtitles{no} % Leave argument "no" if all appendix headings stay EMPTY (then no dot is printed after "Appendix A"). If the appendix sections contain a heading then change the argument to "yes".
\appendixstart
\appendix
\section[\appendixname~\thesection]{}
\subsection[\appendixname~\thesubsection]{}
The appendix is an optional section that can contain details and data supplemental to the main text---for example, explanations of experimental details that would disrupt the flow of the main text but nonetheless remain crucial to understanding and reproducing the research shown; figures of replicates for experiments of which representative data are shown in the main text can be added here if brief, or as Supplementary Data. Mathematical proofs of results not central to the paper can be added as an appendix.

\begin{table}[H] 
\caption{This is a table caption.\label{tab5}}
\newcolumntype{C}{>{\centering\arraybackslash}X}
\begin{tabularx}{\textwidth}{CCC}
\toprule
\textbf{Title 1}	& \textbf{Title 2}	& \textbf{Title 3}\\
\midrule
Entry 1		& Data			& Data\\
Entry 2		& Data			& Data\\
\bottomrule
\end{tabularx}
\end{table}

\section[\appendixname~\thesection]{}
All appendix sections must be cited in the main text. In the appendices, Figures, Tables, etc. should be labeled, starting with ``A''---e.g., Figure A1, Figure A2, etc.

\end{comment}
%%%%%%%%%%%%%%%%%%%%%%%%%%%%%%%%%%%%%%%%%%%%%%%%%%%%%%%%%%%%%%%%%%%%%%%%%%%%%%%%%%%%%%%%%%%%%%%%%
% ACRONYMS
%%%%%%%%%%%%%%%%%%%%%%%%%%%%%%%%%%%%%%%%%%%%%%%%%%%%%%%%%%%%%%%%%%%%%%%%%%%%%%%%%%%%%%%%%%%%%%%%%
\section[\appendixname \thesection]{Acronyms}
\label{sec:acronyms}
%
\begin{acronym}
\acro{OWP}{Office of Water Prediction}
\acro{EWS}{early warning system}
\acro{NWM}{National Water Model}
\acro{NOAA}{National Oceanic and Atmospheric Administration}
\acro{NWC}{National Water Center}
\acro{NFIE}{National Flood Interoperability Experiment}
\acro{NWS}{National Water Service}
\acro{NCAR}{National Center for Atmospheric Research}
\acro{WRF-Hydro}{Weather Research and Forecasting Hydro}
\acro{US}{United States}
\acro{USGS}[US Geological Survey]{United States Geological Survey}
\acro{USD}{US Dollar}
\acro{BLE}{Base Level Engineering}
\acro{InSAR}{Interferometric Synthetic Aperture Radar}
\acro{RAS2FIM}{River Analysis System-2-Flood Inundation Map}
\acro{FEMA}{Federal Emergency Management Agency}
\acro{InFRM}{Interagency Flood Risk Management}
\acro{HAND}{Height Above Nearest Drainage}
\acro{TP}{true positive}
\acro{FP}{false positive}
\acro{TN}{true negative}
\acro{FN}{false negative}
\acro{CSI}{critical success index}
\acro{POD}{probability of detection}
\acro{FAR}{false alarm rate}
\acro{3D}{3-Dimensional}
\acro{2D}{2-Dimensional}
\acro{1D}{1-Dimensional}
\acro{3DEP}{3-Dimensional Elevation Program}
\acro{DEM}{digital elevation model}
\acro{NLD}{National Levee Database}
\acro{NHD}{National Hydrography Dataset}
\acro{NED}{National Elevation Dataset}
\acro{NHDPlus}{National Hydrography Dataset Plus}
\acro{NHDPlusV2}{National Hydrography Dataset Plus Version 2}
\acro{NHDPlusHR}{National Hydrography Dataset Plus High Resolution}
\acro{LP}{level path}
\acro{SRC}{synthetic rating curve}
\acro{SWE}{Shallow Water Equations}
\acro{CONUS}{conterminous United States}
\acro{PR}{Puerto Rico}
\acro{HI}{Hawaii}
\acro{V2.1}{Version 2.1}
\acro{V4}{Version 4}
\acro{km}{kilometer}
\acro{cm}{centimeter}
\acro{RMSE}{root mean squared error}
\acro{$km^2$}{square kilometer}
\acro{$m^2$}{square meter}
\acro{m}{meter}
\acro{FIM}{flood inundation map}
\acro{LiDAR}{Light Detection and Ranging}
\acro{PyGFT}{Python GIS Flood Tool}
\acro{GIS}{Geographic Information Systems}
\acro{InSAR}{interferometric synthetic aperture radar}
\acro{AK}{Alaska}
\acro{NGP}{National Geospatial Program}
\acro{TNM}{The National Map}
\acro{Py3DEP}[Python 3DEP]{Python 3-Dimensional Elevation Program}
\acro{RMSE}{root mean squared error}
\acro{SRTM}{Shuttle Radar Topography Mission}
\acro{HEC-RAS}{Hydrologic Engineering Center River Analysis Center}
\acro{NGS}{National Geodetic Survey}
\acro{NDEP}{National Digital Elevation Program}
\acro{year}{yr}
\acro{hydro-conditioning}{hydrological conditioning}
\acro{hydro-conditioned}{hydrologically conditioned}
\acro{hydro-flattening}{hydrological flattening}
\acro{hydro-flattened}{hydrologically conditioned}
\acro{LULC}{land use/land cover}
\end{acronym}
%
%%%%%%%%%%%%%%%%%%%%%%%%%%%%%%%%%%%%%%%%%%
\begin{adjustwidth}{-\extralength}{0cm}
%\printendnotes[custom] % Un-comment to print a list of endnotes

\reftitle{References}

% Please provide either the correct journal abbreviation (e.g. according to the “List of Title Word Abbreviations” http://www.issn.org/services/online-services/access-to-the-ltwa/) or the full name of the journal.
% Citations and References in Supplementary files are permitted provided that they also appear in the reference list here. 

%=====================================
% References, variant A: external bibliography
%=====================================
\bibliography{bibliography/owp_3dep_2022.bib}


% If authors have biography, please use the format below
%\section*{Short Biography of Authors}
%\bio
%{\raisebox{-0.35cm}{\includegraphics[width=3.5cm,height=5.3cm,clip,keepaspectratio]{dependencies/Definitions/author1.pdf}}}
%{\textbf{Firstname Lastname} Biography of first author}
%
%\bio
%{\raisebox{-0.35cm}{\includegraphics[width=3.5cm,height=5.3cm,clip,keepaspectratio]{dependencies/Definitions/author2.jpg}}}
%{\textbf{Firstname Lastname} Biography of second author}

% For the MDPI journals use author-date citation, please follow the formatting guidelines on http://www.mdpi.com/authors/references
% To cite two works by the same author: \citeauthor{ref-journal-1a} (\citeyear{ref-journal-1a}, \citeyear{ref-journal-1b}). This produces: Whittaker (1967, 1975)
% To cite two works by the same author with specific pages: \citeauthor{ref-journal-3a} (\citeyear{ref-journal-3a}, p. 328; \citeyear{ref-journal-3b}, p.475). This produces: Wong (1999, p. 328; 2000, p. 475)

%%%%%%%%%%%%%%%%%%%%%%%%%%%%%%%%%%%%%%%%%%
%% for journal Sci
%\reviewreports{\\
%Reviewer 1 comments and authors’ response\\
%Reviewer 2 comments and authors’ response\\
%Reviewer 3 comments and authors’ response
%}
%%%%%%%%%%%%%%%%%%%%%%%%%%%%%%%%%%%%%%%%%%
\end{adjustwidth}
\end{document}

